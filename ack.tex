%% The following is a directive for TeXShop to indicate the main file
%%!TEX root = diss.tex

\chapter{Acknowledgments}

Attending university long enough to earn a PhD is an unbelievable privilege, and I only got here with an incredible amount of guidance and support.

Most of all, to mom Calisa Pierce, dad Jim Pierce, my sisters Kelsey and Kim, and my grandmother Wilma Steele (Gugs), you've formed all of the best parts of my personality and continually guided me toward this point, and it's impossible to explain how thankful I am for that.

Mom, during my childhood, you worked full time, taught piano lessons, earned Master and Doctoral degrees, played organ every Sunday, consoled us through every difficulty, and engaged us in as many extracurricular activities as we were willing to take on.
I cannot imagine a better role model of work ethic, open-mindedness, and devotion. I hope I picked a few things up. 

Dad, I've watched you solve an insane diversity of problems without hesitation, from easy ones, like worn out brake pads or blown engines, to hard ones, like burnt eyes and broken arms.
By age 10 I must have helped with copper plumbing, woodwork, car repair, home wiring, and tree felling.
I had unintentional lessons in water currents, fish habits, electrons, engines, wings, hellgrammites, heat transfer, and radio waves. Canoeing the South Branch together taught me more science than school ever did. The overarching message is that phenomena can be understood, and most any problems can be solved, at least with the right tools. It's incredibly useful in a PhD! Thanks for this and so much more.

Kelsey and Kim, my earliest memories are sitting in your bedroom floor struggling through your books, asking for help every couple of paragraphs. You made me good at reading and taught me to enjoy it, and this is the basis of everything. Kim, you were my first role model, and I think of you all the time. Kelsey, I know it goes without saying, but thanks for a lifetime of guidance. You're now, as always, five steps ahead and far more successful. On the one hand, I'm deeply proud at your progress through life, with a stable career and now two beautiful kids, but on the other, I'm flatly shocked by it. 
I'll be playing catch up forever!

Gugs, your approach to life is clean and selfless, and you've been so exceptionally reliable and invested in my development, more like a second mother than a typical grandmother. I'll never be able to thank you enough!
It's always my goal to take up your 4am wakeups and completely measured responses to all of life's challenges. I'm making a bit of progress. Let's hope I can get there!

Finally with regard to family, I'd like to thank my great aunt Alice. Thanks so much for paying my undergrad education! We hardly met before you passed, but we're family, and that was enough. It really means a lot.

Next comes education. Thanks to my early teachers and mentors who put in the extra care to keep me on track and shield me from harm, Jonathan Ramey, Linda Afzhalirhad, Brandon Willard, Linda Mendez, Marty Ojeda, Phyllis Adkins, and Red Hamlin among many others.
Thanks also to the later ones at Southern WV Community College (SWVCC) and West Virginia University (WVU) whose lessons fostered my way of thinking and provided me with many of the tools I used for the research in this thesis.
At SWVCC, I'd like to thank especially Mindy Saunders, Tex Wood, Anne Klein, Charles Wood, Don Saunders, and George Trimble. Mindy in particular transformed my view of the world into one based on mathematics with her incredibly effective teaching, Anne dismantled it into chemistry, and Tex reconstructed it into artistic interpretation, which will always transcend science. Everyone encouraged me to continue my education. You all made SWVCC a transformation and an unbelievable privilege.

At WVU, I'd like to thank Alan Bristow, Tudor Stanescu, and Leo Golubovic. Alan and Tudor, thanks you for providing my first opportunities in research. (Sorry I was bad at it!) Leo, you are the key person in my development as a researcher. You inspired me to learn mechanics, statistical physics, and mathematical physics at a high level, and I can't thank you enough. Thanks to you, I read Mathews, Landau, and Goldstein, and I saw that mechanics was really what I wanted to do. Every time I solve a challenging problem with some Bessel functions or contour integrals, I think ``Dr. Golubovic would like this one." I wish I had taken more advantage of the opportunity to speak to you outside of classes. I'll feel deeply successful if I can one day become as inspiring an educator as you are. I hope we can chat in Morgantown sometime!

To my extended cohort at WVU physics, Payne, Megat, Collins, Scott, Dustin, Rice, Wolfe, Samet, Craig, April, Evan, Larry, Robert, Stephen, Rick, Matt, Caleb, John, and all of you others, thanks for the collaboration and the hilarious times. I'll never forget Scott's quaternions and borderline quantum consciousness beliefs, Samet's olympiad problems, or Megat's power tower magic (which still blows my mind). I have yet to experience teamwork and shared exploration like we had in the undergrad lounge. I hope the lounge is still happening, that its residents by some miracle still have 24 hour access into the building, that some barefoot Megat-like character is still gaming in the corner, and that they have some Andrew Rice imitator to flip some cars for em (mike jones). May that live on forever. Because of you all, WVU was magic for me.

Now we come to Marwan Hassan. About five years ago now I wandered into Marwan's office saying I liked rivers and wanted to describe sediment transport with statistical physics. I realize now when I said ``statistical physics" Marwan's eyes lit up with thoughts of Einstein, Nakagawa, Yano, Hubbell, Crickmore, and so many other names I now know well. I was lucky to use the right trigger words.
Marwan, thanks for taking the risk and taking me on from outside the field, for all of the guidance, for loaning me so many papers and books, giving me freedom to explore ideas, and most of all for showing me that it's ok to be human in science. At this point I also want to thank Yusuf for trying your patience to its limits over a long enough period so that things didn't seem so bad once I came along for this short time together.

My friends in geography have also been great. Thanks to the whole extended research family: Shawn, Matteo, Conor, Nisreen, Yinlue, Alex, Kyle, Leo, Dave, Tobias, Katie, Maria, Elli, David, Jiamei, Xingyu, Niannian, and Emma, plus the outsiders, Will, Dave, Rose, and Anya. In particular, thanks to Shawn for all of the shared speculations, and Matteo and Conor -- friendships forged through suffering! We'll have to keep in touch and transform our shared venting sessions into well-rounded friendships and productive research collaborations.

Thank you Rui Ferreira and Brett Eaton!
Rui, you welcomed me to Portugal, offered me guidance in the laboratory, and educated me on turbulence and hydraulics. Your kindness, deep fluid and granular physics knowledge, and wide-ranging interests beyond physical science are inspiring. Thanks for the thoughtful comments and advice on my PhD. I'm not forgetting what I owe you! But I am impressed at your concise way of speaking and unbounded kindness.
Brett, you were the first person to welcome me to UBC, and I'll always appreciate your effort to teach me how to write, interpret flume experiments, and understand modelling in light of the real world and its complexity. Although I'm not sure if I understood your lessons as well as you would have liked, I still have a list of your favorite books on my computer, and they're a first task for as soon as I can read something besides articles! Maybe they'll bring me closer to your way of thinking.

I'd be amiss not to mention David Furbish. Before I started my PhD, when I was still exclusively a physics kid, I attended Furbish's colloquium in Geography at Marwan's invitation. At the end I asked a vague question about whether Langevin equations might be useful for describing sediment transport, and the answer was ``absolutely yes". Now I have a thesis on it. Later on, David recruited me into a lovely series of meetings where we discussed gases with adhesion, moon craters, entropy, soft matter, martian hillslopes, the social repercussions of disinformation, and a whole range of questions in natural philosophy, pondering implications of stochastic landscape evolution, comparing probability to mass, momentum, and energy, and evaluating fringe ideas in cosmology. These are my people! David, Rachel, Sarah W, Sarah Z, Nakul, Tyler, Erika, and Shawn, thanks sincerely for letting me on board the spacecraft.

Among UBC Geographers, thanks especially to Nina Hewitt! It was always a pleasure and sometimes a sigh of relief to TA your courses, which are so uniquely organized and student-focused. UBC Geography is a much greater department with you in it. I'm so thankful to have been a part of your process to transform undergrads into capable scientists. I'll be happy if I can teach like you one day!

Finally, Mary! My PhD has not been great in terms of work-life balance. Thanks for sticking it out, putting up with my late bedtimes, my early wakeups, and all of those social events I skipped. It's time! I'm done. The pandemic also seems to be winding down. We can finally get to Aus! I'm ready for an extensive tour of the gap and some beautiful Queensland beaches with your family, featuring SPF110 and an incredibly giant umbrella. After that, we're a step closer to settling once and for all whether our cat will be white or orange. I'm leaning toward white again.

Completing a PhD looks like a personal achievement, but it's really not.
It's an expression of 30 years of influence and guidance from other people at every stage, my parents, sisters, grandmother, and a huge network of teachers and mentors.
I'm kinda tired because it's the end, so it's natural to drift into daydreams of Watoga and Williams River with my family, reading encyclopedias in Gugs' living room, lining up just left of center with my dad and bracing on the paddle for the hit, calm discussions with Johnathan Ramey about my education, the glowing fireflies under my great grandpa's tree, ``cross-up, cross-up, squeeze" in Mindy's class, or laughing about a broken watermelon at Kumbrabow with my sisters. It's easy too to shift years later, flipping on the light in the physics lounge at 3am and surprising Megat who was asleep on the couch, or finishing a problem set with Payne and snapping the book shut, discussing Tsujimoto with Marwan, or roaming through Kitsilano with Mary at the edge of dark. The PhD is all of it! Thank you all for every part of it. University is finished.




