%% The following is a directive for TeXShop to indicate the main file
%%!TEX root = diss.tex

\chapter{Acknowledgments}

Attending university long enough to earn a PhD is an unbelievable privilege, and I only had access to it thanks to an incredible amount of guidance and support.

Most of all, to mom Calisa Pierce, dad Jim Pierce, my sisters Kelsey and Kim, and my grandmother Wilma (Gugs) Steele, you've formed all of the best parts of my personality and continually guided me toward this point, and it's impossible to explain how thankful I am for that.

Mom, during my childhood, you worked full time, taught piano lessons, earned Master and Doctoral degrees, played organ every Sunday, consoled us through every difficulty, and engaged us in as many extracurricular activities as we were willing to take on.
I cannot imagine a better role model of work ethic, open-mindedness, and devotion. I hope I picked a few things up!

Dad, in my childhood I watched you solve an insane diversity of problems without hesitation, from easy ones, like worn out brake pads or stuck chainsaws, to hard ones, like burnt eyes and broken arms.
On canoeing trips I got lessons in water currents, fish habits, tree species, electrons, engines, wings, hellgrammites, heat transfer, world history, radio waves, and probably everything else I could think of. Years of free science lessons! You taught me more about science and problem solving than school ever did. "Any problem can be fixed with the right tools" is basically the Jim Pierce philosophy. It it's incredibly useful in a PhD! Thanks for this and so much more.

Kelsey and Kim, my earliest memories are sitting in your bedroom floor struggling through your books, asking for help every couple of paragraphs. You made me good at reading and taught me to enjoy it, and this is the basis of everything. Kim, you were my first role model, and I think of you all the time. Kelsey, I know it goes without saying, but thanks for a lifetime of guidance. You're now, as always (all state chorus, math field day, spelling bees, piano skills, grades, . . . ), five steps ahead and far more successful. On the one hand, I'm deeply proud at your progress through life, with a stable career and now two beautiful kids, but on the other, I'm flatly shocked by it.
I'll be playing catch up forever!

Gugs, you're easily the most level-headed, good-natured, and reliable person I've ever had contact with, and to think I was lucky enough to be about half raised by you! You've been my second mother, my counsellor, at times my daily chauffeur, and my best role model of grace, consistency, and even-handedness. I'll never be able to thank you enough! One day I'll finally take on your 5am wakeups and figure out how to copycat your completely measured responses to all of life's challenges.

Finally with regard to family, I'd like to thank my great aunt Alice. Thanks so much for paying my undergrad education! We hardly met before you passed, but we're family, and that was enough. It really means a lot.

Next comes the mentors in education. Thanks to my early teachers who put in the extra care to keep me on track, Jonathan Ramey, Damon Spurlock, Linda Mendez, Linda Afzhalirhad, Phyllis Adkins, Brandon Willard, Marty Ojeda, Larry Wolford, and Red Hamlin among many others.
	
Thanks also to my professors at Southern WV Community College (SWVCC) and West Virginia University (WVU) who fostered my way of thinking and provided many of the tools I used for the research in this thesis.
At SWVCC, I'd like to thank especially Mindy Saunders, Tex Wood, Anne Klein, Charles Wood, Don Saunders, and George Trimble. Mindy built my math and physics foundation with her incredibly effective teaching, Anne showed me there's invisible order in everything, and Tex demonstrated that art will always lead science. Everyone encouraged me to continue my education. SWVCC was a personal transformation and a great privilege. 

At WVU, I'd like to thank Alan Bristow, Tudor Stanescu, and Leo Golubovic in particular. Alan and Tudor, thanks you for providing my first opportunities in research. (Sorry I was bad at it!) Leo, you are the key person in my development as a researcher. Your ability as an educator and a motivator is incredible. Every time I solve a challenging problem with some Bessel functions or contour integrals, I think ``Dr. Golubovic would like this one!" I wish I had taken more advantage of the opportunity to speak with you outside of classes. I'll be successful if I can one day be as inspiring to someone as you were to me.

To my extended cohort at WVU physics, Payne, Megat, Collins, Scott, Dustin, Rice, Wolfe, Samet, Craig, April, Evan, Larry, Robert, Stephen, Rick, Matt, Caleb, John, and all of you others, thanks for the collaboration and the hilarious times. I'll never forget Scott's quaternions, Samet's olympiad problems, or Megat's power tower magic (which still blows my mind). I have yet to experience teamwork and mutual exploration like we had in the undergrad lounge. Math and physics are fun! I hope the lounge is still happening, that its residents by some miracle still have 24 hour access into the building, that some barefoot Megat-like character is still gaming in the corner, that they have some Andrew Rice imitator to flip some cars for em (mike jones), and that an esteemed graduate student like Payne still finds time to stoop to the lounge-residents' level. May the lounge live on forever, in Dirac's name, amen.

Now we come to Marwan Hassan. About five years ago now I wandered into Marwan's office saying I liked rivers and wanted to describe sediment transport with statistical physics. I realize now when I said ``statistical physics" Marwan's eyes lit up with visions of Einstein, Nakagawa, Yano, and so many other names I now know well. I was lucky to use the right trigger words.
Marwan, thanks for taking the risk on me, for all of the guidance, for lending me hundreds of books and papers, for giving me freedom to explore ideas, and most of all for showing me that it's ok (and secretly expected) to be human in science. 

My friends in geography have also been great. Thanks to the whole extended research family: Shawn, Matteo, Conor, Nisreen, Yinlue, Alex, Kyle, Leo, Dave, Tobias, Katie, Maria, Elli, Jiamei, Xingyu, Niannian, and Emma, plus the outsider Eaton-ites, Will, Dave, Rose, and Anya. In particular thanks to Shawn for all of the shared speculations, and Matteo and Conor -- friendships forged through suffering! We'll have to keep in touch and form productive research collaborations once Matteo comes back from retirement.

Thank you Rui Ferreira and Brett Eaton!
Rui, you welcomed me to Portugal, offered me guidance in the laboratory, and educated me on turbulence and hydraulics. Your kindness, physics knowledge, and wide-ranging interests beyond physical science are inspiring. Thanks for the thoughtful comments and advice on my PhD. I'm not forgetting what I owe you! I will always be impressed at your concise way of speaking and unbounded kindness.
Brett, you were the first person to welcome me to UBC, and I'll always appreciate your effort to teach me how to write, interpret flume experiments, and understand simplified modelling in light of real world complexity. Although I'm not sure if I understood your lessons as well as you would have liked, I still have a list of your favorite books on my computer, and they're a first task for as soon as I'm done. Maybe they'll bring me closer to your way of thinking!

I'd be amiss not to mention David Furbish. Before I started my PhD, when I was still exclusively a physics kid, I attended Furbish's colloquium in Geography at Marwan's invitation. At the end I asked a vague question about whether Langevin equations might be useful for describing sediment transport, and the answer was ``absolutely yes". Now I have a thesis on it. David later recruited me into a lovely series of meetings where we discussed gases with adhesion, moon craters, entropy, soft matter, martian hillslopes, the social impacts of disinformation, and a whole range of questions in natural philosophy from stochastic landscape evolution, to probability as a physical quantity like energy and momentum and fringe ideas in cosmology. These are my people! David, Rachel, Sarah W, Sarah Z, Nakul, Tyler, Erika, and Shawn, thanks sincerely for letting me board the spaceship.

Among UBC Geographers I'd like to thank Nina Hewitt! It was always a pleasure and sometimes a sigh of relief to TA your courses, which are so uniquely organized and student-focused. UBC Geography is a much greater department with you in it. I'm so thankful to have been a part of your process transforming undergrads into scientists. I'll be happy if I can teach like you one day!

Finally, Mary! My PhD has not been great in terms of work-life balance. Thanks for sticking it out, putting up with my late bedtimes, my early wakeups, the rambly science jargon, and all of those social events I skipped. It's time! I'm done. The pandemic also seems to be winding down. We can finally get to Aus! I'm ready for an extensive tour of the gap and some beautiful Queensland beaches with your family, featuring SPF110 and an incredibly giant umbrella. I hope to take up still lives with Marianne's coaching. After that, we're a step closer to settling once and for all what colour cat we're gonna get. I'm personally now leaning toward white.

Completing a PhD looks like a personal achievement, but it's really not.
It's an expression of 30 years of influence and guidance from other people at every stage, my parents, my sisters, my grandmother, and a huge network of teachers and mentors.
I'm kinda tired because it's the end, so it's natural to drift on to recollections of my favorite times in life, Watoga and Williams River with my family, encyclopedias in Gugs' living room, watching eagles over the South Branch with dad, discussions with Johnathan Ramey about my education, the glowing fireflies under my great grandpa's tree, ``cross-up, cross-up, squeeze" in Mindy's calc class, or laughter about stolen watermelons with my sisters. I can skip years later and flip on the light in the physics lounge to find Megat asleep on the couch, visit Gugs on a trip back home, discuss Tsujimoto with Marwan, or walk through Kitsilano with Mary at the edge of dark. The PhD is all of it! Thank you all for every part of it.




