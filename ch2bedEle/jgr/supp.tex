\documentclass[11pt]{article}
% General document formatting
\usepackage[margin=0.75in]{geometry}
\usepackage[parfill]{parskip}
\usepackage[utf8]{inputenc}
\usepackage{subfig}         % side-by-side figures 
% Related to math
\usepackage{amsmath,amssymb,amsfonts,amsthm}
\usepackage{graphicx}
\usepackage{natbib}
\usepackage{titling}
\usepackage{hyperref}
\usepackage{wrapfig}
\usepackage{algpseudocode}
\usepackage{mathtools} % for colon equals 

\usepackage{booktabs} % for wrapping tabulars in accord with
\bibliographystyle{agu}
\setlength{\droptitle}{-5em}   % This is your set screw

%\usepackage[math]{kurier}
\newcommand\be{\begin{equation}} % shortcut to start eq envs 
\newcommand\ee{\end{equation}}   % shortcut to end eq envs
\newcommand\ol{\overline}        % shortcut to draw overline 
\newcommand\bra{\langle}
\newcommand\ket{\rangle}
\newcommand\El{\mathcal{L}}
\newcommand\tg{\tilde{g}}
\newcommand\tG{\tilde{G}}
\date{} % make the date field blank




% for pseudo code -- example from https://tex.stackexchange.com/questions/163768/write-pseudo-code-in-latex
\makeatletter
\def\BState{\State\hskip-\ALG@thistlm}
\makeatother

\begin{document}
\title{Supplementary Material for ``Joint stochastic bedload transport and bed elevation model: variance regulation and power-law rests" by J. Kevin Pierce and Marwan A. Hassan}
\maketitle

\section{Numerical simulation algorithm}

The Gillespie Stochastic Simulation Algorithm (SSA) generates exact realizations of a Markov random process from a sequence of random numbers.
It was originally developed for chemical physics by \citet{Gillespie1977} and is reviewed in \citet{Gillespie1992} and \citet{Gillespie2007}.
The SSA hinges on the defining property of a Markov process. When the transition rates from one state to another are not dependent on the distant past, the process is Markov \citep[e.g.,][]{Cox1965}.
In the following sections, we begin by demonstrating the time intervals $\tau$ between subsequent transitions (i.e., the intervals between transition times) are exponentially distributed within the model we develop in the main text. Then we describe the SSA as a consequence of this property.

\subsection{Times between transitions of any kind}
Our joint stochastic description of bedload and bed elevation changes is characterized by a set of states $(n,m)$ where $n$ and $m$ are positive integers. 
Our description involves four possible transitions (migration in, entrainment, deposition, migration out) with rates given in equations (2-5) in the main text.
From the state $(n,m)$, the rate (probability per unit time) for any transition to occur is the sum over all possibilities:
\begin{multline} A(n,m) = R_{MI}(n+1,m|n,m) + R_E(n+1,m-1|n,m) \\+ R_D(n-1,m+1|n,m) + R_{MO}(n-1,m|n,m).\end{multline}
Using this, the probability that no transition occurs from the state $(n,m)$ in a small time interval $\delta \tau$ is $1-A(n,m)\delta \tau$. If we denote by $Q(\tau|n,m)$ the probability density that a transition of any kind occurs from the state $(n,m)$ after a time $\tau$, we can express the probability density that a transition happens after a slightly larger time $\tau + \delta \tau$ as 
\be Q(\tau+\delta \tau|n,m) = \big[1-A(n,m)\delta \tau\big]Q(\tau|n,m).\ee
This equation invokes the Markov property, since it does not involve the past history of states $(n,m)$. Taking $\delta\tau \rightarrow 0 $ we find the master equation $\frac{d}{d\tau}Q(\tau|n,m) = -A(n,m)Q(\tau|n,m)$, from which we conclude the time $\tau$ between subsequent transitions is distributed as 
\be Q(\tau|n,m) = A(n,m)e^{-A(n,m)\tau}. \label{eq:exp}\ee
Therefore we have shown the time $\tau$ to the next transition from a state $(n,m)$ is exponentially distributed with mean value $\bar{\tau} = 1/A(n,m).$ In deriving this result, we used the normalization condition $1 = \int_0^\infty Q(\tau|n,m)d\tau.$

This result implies if the stochastic process transitioned to the state $(n,m)$ at a time $t$, the next transition will occur at a time $t+\tau$ with $\tau$ a random variable drawn from the exponential distribution \ref{eq:exp}.
Therefore, we can determine the times between subsequent transitions by drawing exponentially distributed random numbers.

\subsection{Selection of transitions that occur}

So far, we have determined how to step the time from one transition to the next, but we have not specified how to step the state variables $n$ and $m$ at each transition time. 
That is, we have not specified the type of transitions that occur at each time step.
Intuitively, this will depend on the relative magnitudes of the rates from equations (2-5) in the main text: the transition with the highest rate is most likely to occur. This is formalized by generating the ratios
\be S = \Bigg\{\frac{R_{MI}(n+1,m|n,m)}{ A(n,m)},\frac{R_E(n+1,m-1|n,m)}{ A(n,m)},\frac{R_D(n-1,m+1|n,m)}{ A(n,m)},\frac{R_{MO}(n-1,m|n,m)}{ A(n,m)}\Bigg\}. \label{eq:rel}\ee
By construction, $\text{sum}\{S\}=1$.
By forming cumulative sums of the four ratios, we partition the unit interval $[0,1]$ into four chunks, each associated with a transition -- either migration in, entrainment, deposition, or migration out. The transitions with the highest rates have the largest associated chunks. To randomly select the transition that occurs at a transition time, we draw a random number on $[0,1]$ and find which chunk it falls in.

In summary, to step the process through a single transition, we draw the time interval to the next transition from the distribution (\ref{eq:exp}), then draw a uniform random from $[0,1]$ and use it to select the transition that occurs from the cumulative sum of the ratios in (\ref{eq:rel}). The SSA simply iterates this random number generation/selection process to generate exact realizations of the stochastic process: these are series of $n$ and $m$ through time from which we can calculate any statistics of interest.

\subsection{Pseudo code for the Gillespie SSA}

To initialize a simulation, we specify the initial conditions $n_0$ and $m_0$, the model parameters for use in equations (2-7) in the main paper, and the desired simulation duration $t_\text{max}$. The SSA uses these inputs to generate time series of $n$ and $m$ as follows:

\rule{\linewidth}{1pt}

\begin{algorithmic}
	\State $t = 0$ 
	\State $n = n_0$\Comment{Set the initial state $(n_0,m_0)$}
	\State $m =  m_0$

	\While{$t<t_{\text{max}}$;}{	\Comment{Simulation will go until $t$ surpasses $t_\text{max}$}
	\State record $(n,m,t)$ \Comment{Build time series of $n$ and $m$}		
	\State draw $\tau$ from eq. (\ref{eq:exp}) \Comment{Select time to next transition}
	\State $t \coloneqq t+\tau$
	\State draw a random number $r$ in $[0,1]$ \Comment{Select type of transition that occurs}
	\State compute the ratios $r_1,r_2,r_3,r_4$ in eq. (\ref{eq:rel})
	\State form the cumulative sums $r_i = \sum_{1\leq j\leq i}r_j$ 
	\Comment{Now enact the transition}
	\If {$0\leq r<r_1$}	\Comment{Migration in}
	\State $n \coloneqq n+1$
	\ElsIf {$r_1\leq r < r_2$} 	\Comment{Entrainment}
	\State $n \coloneqq n + 1$
	\State $m \coloneqq m-1 $
	\ElsIf {$r_2\leq r < r_3$} 	\Comment{Deposition}
	\State $ n \coloneqq n - 1$
	\State $ m \coloneqq m + 1$
	\ElsIf {$r_3 \leq r \leq 1$} \Comment{ Migration out}
	\State $ n \coloneqq n-1$
	\EndIf
}
	\EndWhile
\end{algorithmic}

\section{Approximate solutions of the Master equation}

\subsection{Mean field solution of particle activity}

By assuming the dynamics of the particle activity are totally independent of the bed elevation and summing equation (8) of the manuscript over all values of $m$, we obtain the mean field equation for the particle activity: 
\be 0 = \nu A(n-1) + [\lambda + \mu(n-1)]A(n-1) + \sigma(n+1)A(n+1)+\gamma(n+1)A(n+1)-(\nu  + \lambda + \mu n + \sigma n + \gamma n)A(n).\ee
This can be solved by introducing the generating function \citep{Cox1965} $G(x) = \sum_n x^n A(n)$, providing
\be 0 = (\nu+\lambda)(x-1)G + [\mu x^2 + \sigma + \gamma - (\mu + \sigma + \gamma)x]\frac{\partial G}{\partial x},\ee
which is separable and integrates for 
\be G(x) = \Bigg( \frac{\gamma + \sigma -\mu}{\gamma + \sigma - \mu x}\Bigg)^{\frac{\nu + \lambda}{\mu}}\ee
after applying the normalization condition $G(1)=1$.
From the definition of $G$ we can determine $A(n) = \frac{1}{n!} \frac{d^nG}{dx^n}|_{x=0}$, giving the negative binomial distribution of particle activity demonstrated by \citet{Ancey2008} and stated in equation (10) of the manuscript.

\subsection{Mean field solution of bed elevations}

From the negative binomial distribution we have $\langle n |m \rangle = \langle n \rangle$, so equation (9) provides 
\begin{multline} 0 = [\lambda + \mu \langle n \rangle][1+\kappa(m+1)]M(m+1) + \sigma \langle n \rangle [1-\kappa(m-1)]M(m-1) \\- \{[\lambda + \mu \langle n \rangle](1+\kappa m) + \sigma \langle n \rangle (1-\kappa m) \}M(m). \end{multline}
Identifying $E=\lambda + \mu \langle n \rangle$ and $D = \sigma \langle n \rangle$ and incorporating $E=D$ gives equation (11) of the manuscript
\be 0 = [1+\kappa(m+1)]M(m+1) + [1-\kappa(m-1)]M(m-1) - 2M(m).\ee
The \citet{Martin2014} Ornstein-Uhlenbeck model. 

We solve this using the Fokker-Planck expansion \citep{Gardiner1983} that effectively converts this discrete Master equation for $m$ into a diffusion equation for the quasi-continuous variable $z=z_1 m$ This works since $z_1$ is small. Introducing $z$ and writing $\overline{\kappa}=\kappa/z_1$ gives
\be 0 = [1+\overline{\kappa}(z+z_1)]M(z+z_1) + [1-\overline{\kappa}(z-z_1)]M(z-z_1) - 2M(z).\ee
$\overline{\kappa}$ should not depend on $z_1$ since the magnitude of the feedbacks between bed elevations and entrainment and deposition rates depends on elevation changes, and not on the size of grains or the length of the control volume.
Expanding the entire first and second terms to second order $z_1$ provides the Fokker-Planck equation
\be 0 = -2\overline{\kappa}z_1[zM(z)]' + z_1^2 M''(z). \ee
Taking into account that this distribution should be normalizable, $\lim_{z\rightarrow \pm \infty}M(z) = 0$, we find
\be M = M_0e^{-\kappa (z/z_1)^2}\ee
as stated in the manuscript.

\subsection{Closure equation approach for bed elevations}
We set out to determine an approximate relationship to close $\langle n | m \rangle$ in terms of $m$ valid to first order in $\kappa$.
\be \langle n | m \rangle \approx \langle n \rangle  - \kappa c m,\ee
for the mean particle activity conditional to the elevation $m$ into into manuscript equation (9), noting $E=D$, and neglecting terms of $O(\kappa^2)$ provides 
\be 0 \approx \Big[ 1+\kappa(m+1)\Big\{ 1-\frac{\mu c}{E}\Big\}\Big]M(m+1) + \Big[1-\kappa(m-1)\Big\{1+\frac{\sigma c}{E}\Big\}\Big]-\Big[2-\kappa m\Big\{\frac{(\sigma+\mu)c}{E}\Big\}\Big]M(m)
\ee
At this point, we consider $c$ an undetermined positive constant that may depend on the entrainment, migration, and deposition rates.
Taking the Fokker-Planck expansion and requiring the distribution $M(m)$ to vanish at infinity for normalization yields
\be 0 = 2\kappa m \Big( 2 + \frac{(\sigma-\mu)c}{E}\Big)M(m) + \Big\{
\Big[2-\kappa m \frac{(\sigma+\mu)c}{E} \Big]M(m)\Big\}'.\ee
Finally, integrating, expanding to first order in $\kappa$, and exponentiating to solve for $M$, we find
\be M(m) = M_0 \exp\Big\{-\kappa m^2 \Big[1 + \frac{(\sigma-\mu)c}{2E}\Big]\Big\}.\ee
Since the numerical solutions show $\sigma_m^2 = \frac{1}{4\kappa}$, we determine the closure relation
\be \langle n | m \rangle = \langle n \rangle\Big( 1-\frac{2\kappa m}{1-\mu/\sigma}\Big) \ee
corresponding to $c=2E/(\sigma-\mu)$. This is the closure relationship provided in equation (12) of the manuscript that is plotted with the numerical simulations in manuscript figure (4a).

\bibliography{biblio}
\end{document}



