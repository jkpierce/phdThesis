\section{Introduction: birth-death models for the probability distribution of bedload flux}

In river science, a fundamental problem is the determination of the bedload flux, or the rate of downstream movement of bedload grains \citep{Ballio2014}.
Since bedload transport has strong feedback with stream morphology \citep{Church2006, Recking2016}, its prediction is useful to a wide range of environmental considerations. 
These considerations span a wide range, from aquatic habitat restoration to energy production \citep{Kondolf2014, Wohl2015a}. 
Unfortunately, existing approaches to compute the bedload flux are inadequate. 
Predictions regularly deviate by two orders of magnitude from measured values \citep{Gomez1989, Barry2004, Bathurst2007a, Recking2012}. 

Predicting bedload fluxes is challenging because transport is not always well correlated to average characterizations of flow and bed material. 
Local fluxes can range through orders of magnitude as details of turbulent fluctuations and bed organization vary, while average characterizations of flow and sediment remain constant \citep{Sumer2003, Charru2004, Hassan2008, Venditti2017}.
The same turbulence and sediment organization details which correlate with the bedload flux also interact with it. 
Turbulent impulses induce sediment motion \citep{Valyrakis2010, Celik2014, Amir2014, Shih2017}, moving sediment affects turbulent characteristics \citep{Singh2010, Santos2014, Liu2016}, and bedload fluxes modify the stability and arrangement of bed surface grains \citep{Kirchener1990, Charru2004, Hassan2008}.
The bedload flux is in a cyclical feedback with its controls: these controls are the details of turbulence and bed organization \citep{Jerolmack2005}. 

As a result, bedload fluxes exhibit wide fluctuations, even as average characterizations of flow and bed organization remain steady \citep{Ancey2014}.
In the most controlled laboratory experiments, with steady flows driving uniform glass beads under conditions which do not favor bedform development, instantaneous fluxes are often as much as 200\% mean values \citep{Bohm2004, Ancey2008, Heyman2014, Heyman2016}. 
In natural streams, with temporally or spatially varying grain size distributions \citep{Lisle1992, Chen2008}, unsteady flows \citep{Mao2012, FerrerBoix2015}, a variety of dynamic bed surface structures \citep{Hassan2008, Venditti2017}, variable sediment supply \citep{Madej2009, Elgueta2018}, lateral adjustment \citep{Pitlick2013, Redolfi2018}, and migrating bedforms \citep{Gomez1989, Dhont2018} all imparting additional sources of variabilility through their feedbacks with the flux, instantaneous fluxes can be even larger. 
% Iseya1987, Lisle1989, Kuhnle1998

Apparently, bedload fluxes have statistical characteristics.
Einstein was the first to recognize the statistical character of bedload transport \citep{Einstein1937}. 
He understood transport as a random switching between states of motion and rest \citep{Einstein1942, Einstein1950}. 
He called the transition from rest to motion entrainment, and he characterized it with a probability which was linked to extreme events in fluid turbulence \citep{Einstein1949, Einstein1950}. 
He called the transition from motion to rest deposition, and he considered it an implicit function of the bedload flux \citep{Einstein1942, Einstein1950}. 
By convolving over sequences of entrainment and deposition, employing some semi-emprical arguments, Einstein developed a formula for the average bedload flux \citep{Einstein1950}. 

Many investigators have criticized and developed Einstein's ideas \citep{Paintal1971, Yalin1972, Shen1980, Lisle1998, Papanicolaou2002, Sun2000, Ancey2006, Armanini2015}. 
These revisions focus on the more ad hoc elements of Einstein's derivations, and they lend a more mechanistic basis, a firmer mathematical foundation, and more generality to the approach. 
Although all of these authors accepted the statistical character of bedload motion, none developed theories for a probability distribution for the bedload flux. 
This is desirable since its higher moments give unambiguous measurements of the magnitude of bedload fluctuations. 

The extension of the Einstein approach to obtain a probability distribution of the bedload flux, with fluctuations stemming from the random character of entrainment and deposition, was first explored by \citet{Lisle1998, Sun2000}, and it was further developed by \citet{Ancey2006}, and the more recent work I will discuss extensively in this review.   
All of these authors revisited Einstein's ideas from a foundation in the stochastic mathematics which became formalized somewhat after Einstein's work \citep[e.g.][]{Cox1965}, treating the transitions between motion and rest states as random events characterized by probabilities. 
This enabled them to apply the formalism of continuous time Markov processes to derive a probability distribution of the bedload flux, with a mean value which was an improved version of the \citet{Einstein1950} formula, very similar to the revised formula of \citet{Yalin1972}, and a variance which provided an unambiguous prediction of the magnitude of bedload fluctuations. 

Deterministic processes trace the evolution of some set of variables $\{x_1(t),x_2(t),\dots\}$ through time. 
Stochastic processes, in contrast, trace the evolution of a probability distribution for random variables $P_{X_1,X_2,\dots}(x_1,x_2,\dots;t)$ through time. 
The Markov moniker refers to the amount of memory in the process.
If the probability distribution of the variables of interest can be predicted in the future using only distribution functions from the present, the process is Markovian. 
Otherwise, if predicting the future requires knowledge of the entire history, the process is non-Markovian \citep{Cox1965, VanKampen1992}. 
Markov birth-death processes consider the probabilistic creation and annhilation of members of a population \citep{Cox1965, VanKampen1992}.  

To model the bedload flux with such a process, the population considered is the number of moving particles within a control volume over the bed. 
This population is subject to creation by entrainment and annhilation by deposition \citep{Ancey2008, Turowski2009, Heyman2013, Ma2014,  Ancey2014a, Ancey2015}. 
This approach is exciting because Markov birth-death processes are highly studied in physics, chemistry, and population ecology, so there are many extensions readily available \citep{Bailey1968, Cox1965, Pielou1977, VandenBroek2012, VanKampen1992, Gillespie1992, Field2010, Mendez2015}. 
At the same time, birth-death modeling of the bedload flux remains relatively undeveloped: birth-death models are all one-dimensional and designed for a single grain size. 

In this article, I review the literature on Markov birth-death models of the bedload flux.
In the review I assume some familiarity with discrete state continuous time Markov processes, which could be gleaned from reading the appropriate chapter in \citet{Cox1965}. 
These theories are exciting because they describe the mean bedload flux and its fluctuations within a volume \citep{Ancey2006, Ancey2008, Turowski2009}, statistical properties of the flux at a point in space \citep{Heyman2013, Ma2014}, and the spatial and temporal correlations in the bedload flux within a reach of channel \citep{Ancey2014, Ancey2015}. 
All of these considerations are new, and they are unique within the river science literature in that they do not ignore the statistical character of bedload transport. 
They initiate a theoretical framework from which the feedbacks between the arrangement of bed particles, turbulence, and channel morphology which lead to bedload fluctuations can be more carefully studied and understood. 
In reverse, the capacity of bedload models to describe fluctuations provides an additional benchmark against which to judge models, a task which is notably difficult \citep{Iverson2013}. 

One of the first birth-death models of the bedload flux I will review is \citet{Ancey2006}: these authors generalized \citet{Einstein1950} using a birth-death framework to obtain a statistical distribution of the bedload flux within a control volume. 
They found the fluctuations predicted by the Einstein-like model were not large enough to describe experimental data.  
In order to match the wide fluctuations of experimental data, \citet{Ancey2008} reworked their earlier control volume model to include a collective entrainment effect.
Collective entrainment is a mathematical term charaterizing the correlations between moving grains due to turbulence, collisions, or granular avalanche effects.

The \citet{Ancey2008} model has been generalized and applied in a handful of followup works. 
\citet{Turowski2009} generalized it to include limited sediment availability. 
\citet{Heyman2013} and \citet{Ma2014} studied how the \citet{Ancey2008} model could be interpreted to describe the statistics of the bedload flux at a fixed point in space, rather than just a control volume like that considered by \citet{Einstein1950}. 
Control volumes are less desirable because transport is practically easier to measure at a point in space than it is within a control volume. 
  
\citet{Ancey2014, Ancey2015} generalized the \citet{Ancey2008} model in order to understand spatial characteristics of bedload transport. 
They considered a reach of channel as an array of adjacent control volumes (cells), and they coupled the \citet{Ancey2008} model between adjacent cells. 
They mapped this generalized birth-death model for the bedload flux distributed over an array of control volumes onto an advection diffusion equation for the bedload flux. 
Some reprecussions of this model were further explored in context of experimental data by \citet{Heyman2014}. 

All of these works contain many simplifying assumptions, and many of these were carefully highlighted by their authors. 
These studies were selected for this review because, taken together, they define a research trajectory which can be extrapolated to suggest future research topics. 
To be sure, there is a long road ahead before Markov models can accomodate all of the complexities of natural river channels, some of which I've mentioned. 
The goal of this review is to indicate this trajectory in order to motivate future research: this is pursued in sections \ref{sec:background} and \ref{sec:review}. 
I discuss some of the future research topics I believe the trajectory points out in section \ref{sec:scope}, and I develop a few of these ideas myself in \ref{sec:extension}. 
Along the way, I have somewhat changed the mathematical treatment and the notation from some of the original papers in order to highlight the historical progression between each work, and to illustrate that these works all share the same theme.  

My own extensions concern: (1) birth-death modeling of the fractional transport rates of a sediment mixture -- earlier works concentrate on a single grain size \citep{Ancey2006,Ancey2008,Turowski2009}; (2) the bedload flux of a single grain size when bed elevations vary, modifying entrainment and deposition characteristics -- earlier works concentrate on a flat bed with no bedforms \citep{Ancey2006,Ancey2008,Turowski2009}; and (3) a model for bedload flux including two-dimensional diffusion -- earlier works concetrate on one dimensional diffusion \citep{Ancey2014,Ancey2015,Heyman2014}. 
Once these extensions are presented, in section \ref{sec:conclusion} I recap the literature review, my identifications of underlying assumptions and future research directions, and my own extended Markov birth-death models of the bedload flux.  
