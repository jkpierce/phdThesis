- bedload flux is a fundamental problem
- bedload flux is the rate of downstream movement of sediment grains 
- bedload transport has strong feedback with stream morphology, so it matters
- matters to lots of stuff: aquatic habitat and energy
- unfortunately, existing approaches are inadequate 

- approaches are inadequate because average characterizations of flow and sediment 
do not always correlate with the bedload rate 
- bedload rates can span across orders of magnitude as details of turbulence and bed organization vary, while average characterizations remain constant

- The same local details which correlate with the bedload flux are modified by it. 
- Moving sediment affects turbulent characteristics
- Bedload fluxes modify the stability and arrangement of particles on the bed surface. 
- Thus the bedload flux is in cyclic feedback with the details of turbulence and bed material organization which it correlates with.  
- Through this feedback, the bedload flux inherits randomness. 

- As a result, bedload fluxes exhibit wide fluctuations of several times mean values, even as average characterizations of flow and sediment remain steady. 
- These fluctuations occur in the most controlled laboratory experiments, with uniform size glass beads and steady state conditions 
- They occur especially in natural streams, where multiple grain sizes, unsteady flows, and variations in sediment supply only exascerbate the feedbacks imparting randomness to the bedload flux -- the linkages between bed material organization, turbulence, and bedload motion. 
- Apparently, bedload fluxes have statistical characteristics. 

- Einstein was the first to recognize the probabilistic character of bedload motion
- He understood bedload motion as a random switching between resting and motion states
- He called the transition from rest to motion entrainment, and he characterized it with a probability which was linked to extreme events in fluid turbulence \citep{Einstein1949, Einstein1950}. 
- He called the transition from motion to rest deposition, and he considered it an implicit function of the bedload flux \citep{Einstein1942, Einstein1950}. 
- By convolving over sequences of entrainment and deposition, employing some semi-empirical arguments along the way, Einstein developed a formula for the average transport rate which was based upon statistics \citep{Einstein1950, Yalin1972, Shen1980, Ancey2006}. 

