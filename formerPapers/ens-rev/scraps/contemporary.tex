\section{Contemporary Markovian sediment transport theories}

%%%%%%%%%%%%%%%%%%%%%%%%%%%%%%%%%%%%%%%%%%%%%%%%%%%%%%%%%%%%%%%%
\subsection*{Sun 1989: a first markov model approach} 
Need to get the reference still 

%%%%%%%%%%%%%%%%%%%%%%%%%%%%%%%%%%%%%%%%%%%%%%%%%%%%%%%%%%%%%%%%
\subsection*{Lisle et al 1998: Generalizing Einstein 1937 to a finite velocity} 
\citep{Lisle1998}

%%%%%%%%%%%%%%%%%%%%%%%%%%%%%%%%%%%%%%%%%%%%%%%%%%%%%%%%%%%%%%%%
\subsection*{Papanicolaou 2002: did a thing}
\citep{Papanicolaou2002}
"characterized the sediment entrainment
process using a two-state continuous-time Markov chain"

%%%%%%%%%%%%%%%%%%%%%%%%%%%%%%%%%%%%%%%%%%%%%%%%%%%%%%%%%%%%%%%%
\subsection*{Ancey et al 2006: Reformulating Einstein 1950 with a more physically based foundation}
\citep{Ancey2006}

\subsection*{Wu and Yang 2004: Quasi-four state model}
\citep{Wu2004a} is the reference. 

%%%%%%%%%%%%%%%%%%%%%%%%%%%%%%%%%%%%%%%%%%%%%%%%%%%%%%%%%%%%%%%%%%%%%%%%%%%%%%%%%%
\subsection*{Ancey et al 2008: Including collective entrainment into Einstein 1950} 
\citep{Ancey2008}

%%%%%%%%%%%%%%%%%%%%%%%%%%%%%%%%%%%%%%%%%%%%%%%%%%%%%%%%%%%%%%%%%%%%%%%%%%%%%%%%%%
\subsection*{Ancey et al 2014: From birth-death to diffusion} 
\citep{Coleman2009, Ancey2014}
\begin{enumerate}
\item Exner 1927 -- the exner equation 
\item Coleman and Nikora 2009 -- the exner equation
\end{enumerate} 

%%%%%%%%%%%%%%%%%%%%%%%%%%%%%%%%%%%%%%%%%%%%%%%%%%%%%%%%%%%%%%%%%%%%%%%%%%%%%%%%%%
\subsection*{Sun et al 2000: model for any size fraction of sediment which is basically Einstein}
\citep{Sun2000}

%%%%%%%%%%%%%%%%%%%%%%%%%%%%%%%%%%%%%%%%%%%%%%%%%%%%%%%%%%%%%%%%%%%%%%%%%%%%%%%%%%
\subsection*{Tsai et al's three-state markov for mixed size sediment} 
\citep{Tsai2013, Tsai2014, Kuai2016}

adopted a three-state
continuous-time Markov chain model to simulate the particle
movement among three different states for uniform and nonuniform
particle size, respectively.

there is a 2014 and 2013 tsai paper, also investigate the 2016 kuai and tsai paper 

"The objective of this paper is to develop a nonhomogeneous
discrete-time 3-state Markov chain model that can efficiently quan-
tify both the bedload and the suspended load discharges under un-
steady flow conditions for mixed size sediment particles"

%%%%%%%%%%%%%%%%%%%%%%%%%%%%%%%%%%%%%%%%%%%%%%%%%%%%%%%%%%%%%%%%
\subsection*{Turowski2009 generalized ancey to bedrock situations with finite supply} 
\citep{Turowski2009}



\subsection*{Armanini 2015: Including entrainment probability and hop length distribution in Einstein formulation} 
\citep{Armanini2015}



\begin{comment} 


Others to consider 


Markovian phase switching for low sediment transport rates 
\begin{itemize}
\item Dey 1999 
\item Dey 2018
\item Einstein et al, 1937
\item Kalinske 1947 
\item Einstein et al 1950 
\item Nakagawa and Tsujimoto [1980] pickup einstein school 
\item van Rijn [1984] pChien, N., and Wan, C. H. (1983). Mechanics of sediment transport.ickup einstein school 
\item Tsujimoto [1991] pickup probability 
\item weeks 1996 assymetrical random walk 
\item Lisle 1998 Sedi xport as birth death process 
\item sun 2000 - mixed size sediment transport as 2 state markov
\item tasi and yang 2013 three-state markov chain model 
\item Ancey et al 2006 - rederivation of einstein from birth-death
\item Ancey et al 2008
\item Ancey et al 2010
\item Ganti 2010  - anomalous diffusion from fractional pdes 
\item Ancey et al 2014 
\item Hu and Guo 2011 -- this is a probability entrainment model which is pretty conventional and not really standout. It contains a lot of good chinese references. Also investigates the concentration of near bed particles. This is not really markovian 
\item Armanini 2014 -- reworking of Einstein 1950 
\item Yalin 1972 / 1977 / 1992-- criticism of Einstein 
\item Armaniini 2018 -- river hydraulics book 
\item Exner 1927 -- the exner equation 
\item Coleman and Nikora 2009 -- the exner equation
\item Ballio 2014 -- def of solid discharge
\item Martin 2014 mean reverting random walk 
\item Fan 2014 random walk from langevin equation
\item Fan 2016 random walk + ad hoc resting times 
\item sun 2015 - an anomalous diffusion model? 
\item Ballio 2018 -- Lagrangian vs Eulerian approach
\item Furbish et al 2016 Jayne Entropy 
\end{itemize} 

wu chen did bayesian updating on an entrainment model 




Housais argues that sediment motion is a continuous transition driven by granular creep, at which point the partition between phases somewhat breaks down. 

\subsection{Stochastic modeling approaches: phenomenological versus mechanistic}

In addition to the division between deterministic (like Bagnold) and stochastic (like Einstein) approaches, another division can be made between the mechanistic and the phenomenological approaches.
Mechanistic approaches rely on underlying physics including fluid and granular properties and their interaction in order to phrase sediment transport processes. 
Phenomenological approaches start from a set of prescribed assumptions with differing amounts of empirical basis. 
Phenomenolgoical approaches include the transport model of Einstein (1950). 
If chaos, the innumerable degrees of freedom, and open system issues guide us toward stochastic approaches, the difficulty of explaining the phenomena of sediment transport apparently encountered when using phenomenological approaches 
The failures of phenomenolgoical approaches include a lack of generalizability, ... . 
(this is all kinda introductory stuff instead)

Investigators have understood the motion of sediment through various simplified conceptual pictures.
Often, a division of sediment motion into phases is made. 
One common scheme is to understand sediment transport as a switching between motion and resting phases. 
This reveals three processes for consideration. These are entrainment, motion, and deposition.
Of course, these are idealized categories. 
At higher transport stages, for example, there must be a breakdown of these three distinct processes. 
They are expected to blur together, or to become less well-defined.
Therefore, most mechanistic-stochastic approaches of interest for this review will be seen to hold only in the limit of low transport stage. 
This is a fundamental limitation of these approaches which we will acknowledge but not attempt to amend. 

We will follow the conceptuatlization of sediment transport as emergent from the coorporation of three distinct processes, as has been assumed by many reasearchers since Einstein, certainly. 
Along the course of the review, it has become clear that many approaches have concentrated on only one of these processes, while others have attempted to integrate or convolve these processes together to arrive at a prediction (or a range of predictions) for the sediment transport rate. 
Therefore, studies of entrainment, motion, and deposition are considered separately in this review. 
Subsequently, approaches to integrate or convolve these separate processes together into a sediment transport rate are considered. 

\subsection{entrainment}

The erosion process of sediment grains from the rest phase to the motion phase is the most well-studied through a mechanistic lens among entrainment, motion, and deposition.  
A majority of this research has concentrated on the determiniation of the so-called incipient motion criterion. 
Investigators have sought the flow condition corresponding to the initial motion of a sediment grain from the bed. 
Historically, investigators have differed on the flow characteristic determining initial motion. 
Shields (1936) contended entrainment was characterised by an average shear stress, and the influence of this idea has been considerable (Mongomery legend of sheilds, Gomez 1991, Montgomery 1997).  
White (1940) understood entrainment as the result of instantaneous shear stress, and this idea was later taken up by many others (Grass, 1971; Paintal 1971; Hofland 2006).
Others have considered incipient motion as controlled by the instantaeous fluid velocity (White 1940; Kalinske 1947; Einstein, 1950). 
A huge set of theories have been developed to the incipient motion of sediment grains. 
The majority of these studies were developed in a mechanistic-deterministic lens (Wiberg and Smith 1987; Bridge and Bennett 1992).
Other investigators contended that the incipient motion criterion could not be determined precisely. 
Since the 2000s, and indeed much earlier, this fact has been quite well supported by experimental evidence. 
A contemporary perspective on the incipient motion criterion problem is that sediment motion is driven by turbulent fluctuations. 
Research into this movement was initiated by the famous work of Kline (1967) revealing large, coherent, turbulent structures in the flow field of a sheared fluid. 
There were several careful investigations into the role of turbulence in sediment transport inspired by Kline's work, including Nelson 1995 and Sumer 2003. 
A large set of investigations has subsequently quantified the role of these turbulent fluctuations in the initiation of sediment motion. 
Sediment motion is best conceptualized as an extreme value approach, most similar to the thinking of Grass, Paintal, Kalinske, Einstein, and others; but in fact differing fundamentally in one regard. 
Instead of instantaneous fluid velocity or shear stress being the best characteristic of initial sediment motion, a collection of modern studies utilizing real-time fluid velocity or pressure measurements seem to indicate that impulse, a time integrated flow characteristic, along with granular geometry, appears to be the best flow indicator of bed particle incipient motion. 
This body of research into impulse was probably opened by Diplas (2008), although it must have been considered much earlier as a possibility. Diplas et al indicated, by measuring the instantaneous velocity just upstream of particles, that an incipient velocity criterion was not present. 
Sometimes, even if the upstream velocity of the fluid to a sediment grain was above the incipient motion criterion, particles would not entrain. 
They determined that an incipient fluid velocity was necessary but not sufficient for incipient sediment motion, and propoesd that instead, sediment motion was the result of a sufficient fluid impulse. 
Many other investigations have followed up on this impulse hypothesis. 
Valykyris, Celik 2010, etc
The contemporary microscale understanding of sediment entrainment most defensible in light of the considerable research into the interrelationship between fluid dynamics and particle motion, is that impulse, or a driving force integrated through time, not instantaneous fluid velocity or shear stress, is the best coorelate with the incipient motion of sediment grains. 
As this perspective has only emerged in the last ten years, one can argue that all previous entrianment models are somewhat invalidated by it within certain ranges of the parameter spaces over which they purportedly apply. 
In acknowledgement of this, we begin a historical review of mechanistic stochastic entrainment theories. 
When this review is complete, an original ammendment of some of these theories to take account of the perspective that a distribution of impulses drives seidment entrainment is presented. 

\begin{enumerate}
\item Einstein 1937
\item Einstein 1950
\item White 1940 
\item Grass 1971
\item Paintal 1971
\item Gessler 1977 
\item Kirchener 1990 
\item Bridge and Bennett 1992
\item Church 1994 
\item Cheng and Chiew 1998
\item Wu and Chou 2003
\item Ma, Xu 2013
\item Papanicolaou 2002
\item Wu and Yang 2004
\item Hofland 2006
\item Valykyris 2011
\item Ling 2012 
argues that impulse based criteria are necessary to work in a grass-like approach 
\item Tregnaghi 2012

\end{enumerate} 


\subsection{motion}

\begin{enumerate}
\item Furbish 2015

\end{enumerate}

\subsection{deposition}
\begin{enumerate}
\item Charru 2004 
\end{enumerate}

\subsection{transport}


\begin{enumerate}
\item Einstein 1950 
\item Paintal 1971
\item Ling 1998
\item sun donahue 2000 
\item Tsai and Yang 2001
\item Martin 2014 
\item Ancey 2006
\item Pelosi 2014
\item Sun 2015 
 
\end{enumerate}
\end{comment} 
