\documentclass[11pt]{article}
% General document formatting
\usepackage[margin=0.75in]{geometry}
\usepackage[parfill]{parskip}
\usepackage[utf8]{inputenc}
\usepackage{subfig}         % side-by-side figures 
% Related to math
\usepackage{amsmath,amssymb,amsfonts,amsthm}
\usepackage{graphicx}
\usepackage{natbib}
\usepackage{titling}
\usepackage{hyperref}
\usepackage{wrapfig}
\usepackage{booktabs} % for wrapping tabulars in accord with
\bibliographystyle{agu}


%\usepackage[math]{kurier}
\newcommand\be{\begin{equation}} % shortcut to start eq envs 
\newcommand\ee{\end{equation}}   % shortcut to end eq envs
\newcommand\ol{\overline}        % shortcut to draw overline 
\newcommand\bra{\langle}
\newcommand\ket{\rangle}
\newcommand\El{\mathcal{L}}
\newcommand\tg{\tilde{g}}
\newcommand\tG{\tilde{G}}
\begin{document}
	
	\title{Replies to reviewers: ``Back to Einstein: How to include sediment burial in bedload diffusion models?''}
	\author{James K. Pierce \& Marwan A. Hassan}
	\maketitle



\section*{Associate Editor}

Both reviewers appreciate the modelling approach but comment on the need to focus more on the physics than the model itself. Some references are missing and in general the authors should take some time to rethink the way the paper is written so that it is clear what this model does with respect to the existing literature and how it fits in it and how does this model advance our understanding of sediment transport. Although I don't expect the main conclusions of the paper to change, it would be good for the authors to have the time to change the writing approach and make sure the overall message of the paper justifies urgent publication.


\section*{Reviewer \#1}
\subsection*{Summary:}
In this work, a CTRW model that explicitly incorporates motion, rest, and burial states is used to explain changing diffusive regimes of gravel tracers in rivers. In particular, different diffusive states are related to the mean duration of particle motion, mean duration of particle rest, and mean duration of particle rest prior to burial. This is a nice example of theoretical stochastic models pointing toward physical explanations of particle transport observations.
This work is appropriate for GRL after some revisions:

\paragraph{Reply:}
Thank you reviewer \#1 for your careful reading and commentary. Your summary shows perfect understanding and we found all of your comments apt and constructive. Therefore, we responded to every comment with changes in the manuscript. We identify three major themes in your comments: (1) you indicated that the manuscript's literature review needed to more carefully outline why we expect three diffusion ranges and what earlier investigations found; (2) you pointed out several oversights --  we used the term ``scale-dependence'' without really defining it, we did not cite the papers of Furbish and coworkers, and we did not specify that we were discussing the transport of a group of tracer grains; and (3) you raised the point that many of the mathematical details included in the manuscript are not essential to the physical story. We responded to (1) and (2) by totally reworking the manuscript's literature review, carefully defining ``scale dependence'' and minimizing its usage, citing all relevant papers by Furbish and coworkers, and emphasizing that our topic is the diffusion of tracers. We have responded to (3) by moving the majority of the mathematical details into a supplement, including all three appendices and a large chunk of section 2. Thank you again for your careful comments that have improved the manuscript.


\subsection*{Specific Comments:}
\begin{enumerate}
\item L. 37 diffusion is induced by transport characteristics of grains, but also the conditions they encounter

\paragraph{Reply:} We have moved this sentence to be the first of the introduction and included the suggested revision.

\item L. 43 Did Einstein really conclude that bedload transport shows normal diffusion? More specifically, no mathematical tools existed that would suggest that it was possible for rest times to not be described by a mean.

\paragraph{Reply:} Yes, Einstein describes normal diffusion of bedload in equation 19 of his 1937 PhD thesis \citep{Einstein1937}.
He also presents flume measurements of the sediment resting time and step length distributions, which he found to be exponential. In his introduction, he carefully describes his realization that the sediment resting time cannot be characterized by its mean value. He used the exponential resting time distribution extensively in his analysis, which is a pioneering application of random walk theory that precedes the classic work of \citet{Montroll1965} by nearly three decades.

\item L . 54 and elsewhere- I think that the use of "scale dependence" in the entire manuscript needs to be clarified. Scale dependence as a problem most commonly refers to model parameters changing with time or space scale - usually this is because the model is incorrectly applied- for example using the diffusion model for a superdiffusive process results in the D coefficient being scale dependent. Here, we have a process whose scaling regime itself changes after crossing characteristic timescales and we need a model that can accommodate this- as demonstrated in this manuscript. This needs to be clarified
\paragraph{Reply:} Thank you for raising this point. We have clarified our use of ``scale dependence'' at L. 58, so we could use it without ambiguity in the rest of the manuscript.

\item L. 54 Rather than the lit review just saying that others have shown different types of scale dependence, please specify what they found. There is comparison to Nikora's 3 scales- but these have different origins than those described in this manuscript. For example the smallest scale includes correlation of velocity in Nikora's work.
\paragraph{Reply:} 

We totally reworked this section to specify what various studies have found. We split the literature review into one paragraph about experimental studies and another about modelling studies. As we discuss after L. 54, at base level Nkora attributed the smallest scale to the smoothness of bedload trajectories in between subsequent resting periods. This timescale is related to the correlation of (virtual) velocity, so there is no essential difference.

\item Section 3 and Table 1. The mathematical solution and description of functions used to solve PDEs is not relevant to the story. I suggest moving much of this section to a supplement and dedicating more of the paper to the sediment transport/geomorphic story. For example, Nikora's global timescale is subdiffusive, while yours is superdiffusive. Why? How your model relates to all the others you mention is more important to the reader and how they should model tracer transport.
\paragraph{Reply:} 
We moved the majority of this information to a supplement as you suggest, and we have added some text to the discussion to clarify these issues. However, we note that the Nikora data was assembled by patching together several datasets, meaning we don't have total faith in these diffusion ranges to begin with. For example, the local range was resolved by numerical simulations, the intermediate range was resolved by the Duck creek data, and the global range was resolved by the \citet{Drake1988} data.

\item Reference to the detailed bedload lab studies and detailed statistical descriptions in a variety of manuscripts by Mark Schmeeckle and David Furbish are completely missing from this manuscript and make statements about transport and resting times not having been studied incorrect. Incorporating their large body of work into this manuscript will strengthen it.
\paragraph{Reply:} We added citations to \citet{Fathel2015}, \cite{Furbish2016}, \citet{Roseberry2012}, and \citet{Furbish2012b} as they directly support our use of exponential models for resting and motion intervals.

\item Mention that what is being modeled is the spreading of a cloud of tracers, rather than bulk transport generally. This will help the reader understand that line 217- all tracers become buried. "All particles" can not be buried by definition- some have to be at the top.
\paragraph{Reply:} We added the requested clarification in many places in the manuscript, perhaps most usefully in L7 (first key point) and L 27 (first sentence of introduction). 

\item Key points- the key points should tell us takeaways... instead of we apply the model... perhaps talk about the characteristic timescales and what they represent. Have more of a physical focus than a model focus.
\paragraph{Reply:} 

We totally revised the key points to address this shortcoming. We sincerely appreciate your careful attention and constructive comments, reviewer \# 1.
\end{enumerate}

\section*{Reviewer \#2}
\subsection*{Summary}
This paper presents a derivation of a continuous time random walk model for bed load particle diffusion that proposes four scaling regimes. As far as I can tell the manuscript is technically correct, though it is not clear who the target audience is or how it advances our knowledge of sediment transport over other bed load diffusion treatments.

\paragraph{Reply:}
We thank you for the careful reading, reviewer \# 2. We have added a clear statement of who the target audience is at L. 24 and also modified the first paragraph to clarify this point. We have modified the work to place its key contribution to bedload diffusion understanding front and center: it is the first model of three diffusion ranges describing the propositions of Nikora. We synthesize the literature to argue that three diffusion ranges is correct in L. 61-93, then we point out that have developed the first analytical model of three diffusion ranges at L. 9, L. 269, and L. 395.  

We believe the key contribution of our paper is that it's the first model deriving three stages of bedload diffusion following the expectation for three set out by \citet{Nikora2001a} and existing experimental data. We'd like to highlight that \citet{Wu2019}, \citet{Wu2019a}, and \citet{Lajeunesse2018} have all highlighted the need for this extension. 

\subsection*{General Comments}
This is a challenging review for me. On one hand the manuscript appears, to my knowledge, technically correct, however I am not sure that I really understand how this advances the field over other recent attempts at this problem. This is another attempt at approximating bed load transport as a diffusion problem and similar to past attempts at the problem it reveals several diffusion regimes for a conceptual set of particles traveling with like sized particles. It is not clear what the correct number of diffusion regimes actually is for bed load (two, three, four)? From my understanding of previous papers (most of which are cited within the manuscript), the number of scaling regimes and the numerical values of the scaling exponents depends on the types of distributions used for the rest and travel durations and the particular sets of equations used within the model. Here they are taken as exponential without justification, I suspect because it makes the derivation simpler? Furbish and colleagues provide numerous examples of what these distributions may be from a mathematical perspective and support them with experimental data. Perhaps it would be useful to consider their advances as the starting point for the model application here. Given the data supported theoretical distributions, would you still derive three (four counting the new longer time constant one) regimes? It is not clear to me how challenging it would be to incorporate their results, but it would provide a stronger rational for the results presented here.

My recommendations for making this manuscript stand apart could take several different routes. Option 1 - edit the manuscript to place the new capabilities of this model front and center. What does this model now allow us to do that we could not derive from past models of this problem (i.e. what does the analytical solution allow us to do moving forward that the many previous numerical solutions can't do). As an example, as a non mathematician I can not tell if the analytical derivation is particulary novel or a standard application of mathematical concepts. Option 2 would be to utilize some form of field of laboratory data to demonstrate that the four scaling regimes are indeed representative of bed load transport rather then being the result of a set of mathematical equations that may represent a simplified conceptual form of a process like bed load transport. These are only suggestions and I am sure there are other routes that could be taken to clarify the impact of the contribution.

It is hard for me to rate the article for above a science category 3 in its current form, however given that it appears correct it is up to the authors to convince the editor that this paper is indeed an advance worthy of rapid publication. The presentation of the manuscript is fine and the writing is easy to follow, the figures are fine as they are but they don't necessarily add much to the paper as is. 


\paragraph{Reply:}
You've presented us a unique perspective on the weak points in the work, and we've worked to concentrate on the key points you've raised and act on them.
We hope you don't mind our summarizing of key points from your general comments and replying to each in turn. 
\begin{enumerate} 
\item The target audience is unclear.
\paragraph{Reply:}
We have added text at L. 24 and between L. 29 and L. 32 to illustrate the target audience and some practical problems this fundamental research lends perspective to.

\item The impact of the work is not obvious.
\paragraph{Reply:}
We have rewritten the paper to emphasize its essential contributions to bedload diffusion understanding. This is the first analytical description of bedload diffusion across the local, intermediate, and global timescales introduced by \citet{Nikora2001a}. The model thus demonstrates the required components for three bedload diffusion ranges: these are (1) the duration of sediment motions, (2) mobile-immobile switching, and (3) a sediment trapping process.
In addition, the work re-frames earlier studies \citep[e.g.,][]{Wu2019,Lisle1998,Lajeunesse2018,Einstein1937} within a comprehensive mathematical formalism built out of a fusion of condensed matter \citep[e.g.,][]{Weiss1994} and chemical physics \citep[e.g.][]{Schmidt2007} ideas. 

\item The expected number of diffusion ranges is unclear.
\paragraph{Reply:}
This is an uncertainty shared by the research community that we set out to address. Your comment indicated we failed to achieve this in the initial manuscript, so we worked to make the points clear in the revised manuscript. Please see L. 61-98. In short, cross comparing experimental studies demonstrates three diffusion ranges, while scale limitations prevent any one experiment from demonstrating three diffusion ranges. Meanwhile, models show all one or two diffusion ranges, but only one Newtonian simulation approach has shown all three \citep{Bialik2012}. 

\item You believe the type of distributions might determine the number of regimes rather than the number of states (i.e., motion, rest, burial).

\paragraph{Reply:}
Random walk models show asymptotic diffusion characteristics that are independent of the functional form of the input distributions, provided light tailed distributions are not swapped for a heavy tailed ones \citep{Weiss1994,Weeks1998}.
In the preliminary stages of this work, we attempted to model Nikora's three diffusion ranges using heavy tailed sojourn times in a two state model. This is actually suggested in the conclusion of \citet{Nikora2001a}, but we determined from numerical simulations that it did not work, and in any case such a model would not be analytically solvable, so we moved on to try incorporating the sediment burial process. In the end, we came to the conclusion that three ranges stem from three states: motion, rest, and burial. We do not believe heavy tailed distributions are sufficient.
We note \citet{Lisle1998}, \citet{Lajeunesse2018}, and \citet{Wu2019} are all two state random walk models showing two ranges of diffusion, whereas our three state model shows three ranges of diffusion.

\item You take issue with the simple assumption of exponential sojourn times.

\paragraph{Reply:}
We have added text to the manuscript and a handful of citations to Dr. Furbish and coworkers to justify our choice of these distributions between L. 179 and L. 185.
In short, the exponential assumption does make the derivation easier because it makes the stochastic model Markovian \citep[e.g.,][]{Cox1965}. However, we emphasize that prevalent models in the literature have simplified even further than we have by neglecting all statistical moments except the first two \citep[e.g.][]{Wu2019}. This leads to highly simplified Gaussian probability distributions of the tracer cloud, in contrast to the less idealized Bessel distributions in the manuscript.

\item You believe we have neglected the works of Furbish and coworkers.
\paragraph{Reply:}
\citet{Furbish2012a}, \citet{Roseberry2012}, \citet{Fathel2015}, and \citet{Furbish2016} all provide direct support for our use of exponential travel and resting time distributions and we have incorporated these citations into the text. Our work is totally aligned with these papers, as they mostly confirm and build up the much earlier findings of \citet{Einstein1937}.

\item You are unsure if changing the distributions from exponential will affect the essential conclusion of three diffusion ranges.
\paragraph{Reply:}
We still expect $3$ $(+1)$ regimes regardless of the distributions chosen, provided they are light tailed and there is a sufficient separation of motion, rest, and burial timescales.

\item You wonder how challenging it would be to incorporate different distributions.
\paragraph{Reply:}
Choosing other light tailed distributions would be quite easy in numerical simulations, but difficult or impossible using analytical methods. 

\item You recommend placing the capabilities of the model front and center to address the new features of the model.
\paragraph{Reply:}
We have added reference to the key contribution of this paper (a first analytical description of Nikora's three diffusion ranges) at L. 9, L. 101, L. 271, and L. 397. 


\item You are unsure of the novelty of the analytical derivation and its relative merits to earlier numerical approaches.
\paragraph{Reply:}
We are unaware of any other model of more than two diffusion ranges in the broader physics literature, while the interdisciplinary impact of anomalous diffusion research has been carefully highlighted \citep[e.g.,][]{Sokolov2012,Metzler2014}. We are fusing ideas from chemical physics \citep[e.g.,][]{Schmidt2007} and condensed matter physics \citep[e.g.,][]{Weiss1994} to describe an open problem in river science, and we have not seen an analysis quite like this in the bedload transport literature. For this reason, we chose to describe our mathematical analysis with sufficient depth to be easily understood by our audience.
Nevertheless, we have acted on your comment and moved a great deal of the mathematical details to a supplement.

\item You suggest field or laboratory data might strengthen the paper.
\paragraph{Reply:}
We have added discussion between L. 61 and L. 73 to indicate why experimental resolution of three diffusion ranges has been so elusive. We do not believe existing technology allows such a resolution. This is part of the reason our model explaining three diffusion ranges from a set of simple assumptions is a useful contribution to the research community.

\end{enumerate}
In summary, we appreciate your careful reading and constructive comments, and we sincerely thank you for your attention toward improving our work. Now we will turn to your specific comments.


\subsection*{Specific Comments}
\begin{enumerate}

\item 
Throughout paper - Single authors seem to have initials cited in text. See N. D. Bradley ln. 32. May be a result of his initials being reported two ways in your references. Likely an error in the reference manager software.

\paragraph{Reply:}
Thanks! We fixed this. It was a Mendeley problem as suggested.

\item 
Ln. 55-63. Is their a physical justification why we should expect three regimes? Why do these studies, taken collectively, produce conflicting results? If this paper could synthesize this literature and demonstrate (a) why three regimes is correct and (b) why the other studies do not produce a consensus picture of the problem that would be a very useful contribution.

\paragraph{Reply:}
The present work attempts to convince readers the necessary ingredients of three stage diffusion are (1) a motion interval, (2) a mobile-immobile switching, and (3) a trapping process. We believe the introduction now outlines why we expect three ranges in line with Nikora, despite the incomplete physical motivation provided in the early Nikora papers and highlighted after L. 54. 

\item 
Ln. 151. What is the justification for the exponential rest distribution functions? Likewise please provide a justification for the exponential durations of a motion in line 157-158.

\paragraph{Reply:}
 

Similarly \citet{Furbish2016}, \citet{Furbish2017}, \citet{Ancey2006}, \citet{Ancey2008}, \citet{Wu2019}, \citet{Lisle1998}, \citet{Lajeunesse2018} have all used, measured, or derived exponential motion durations.
We have discussed this carefully between L. 179 and L. 186. 


\item 
Figure 2. These timescales seem a bit short for bed load transport. The geomorphic timescale is on the order of 8 hrs to ~10 days (less than one snowmelt season in the Rockies). I understand that this is for parameters derived within a small scale flume, but even a recirculating flume might run well past these timescales.

\paragraph{Reply:}
The geomorphic timescale has $T_G \propto \sqrt{T_b}$, where $T_b = 1/\kappa$ is the mean time required for sediment resting on the surface to become buried. In figure 2 we arbitrarily chose $T_b = 2$hrs, implying the geomorphic timescale of order $8$hrs to $10$days you mention. Were we to increase $T_b$ to $200$hrs, the geomorphic timescale would increase to $80$hrs to $100$ days. A burial timescale of $200$hrs still seems reasonable enough compared to field studies of vertical exchange of sediment such as \citet{Ferguson2002} and \citet{Haschenburger2011}. We are unsure of a realistic value for this timescale, as no measurements of it exist, so we highlight this uncertainty at L. 386 as a topic for future research. 

\item 
Ln. 245. Is there a physical explanation for the super ballistic scaling within the local region? This was also observed within the recent work of Lajeunesse et al. (2018 - Advection and dispersion of bed load tracers, Earth Surface Dynamics), though they suggest that only two regimes are necessary.

\paragraph{Reply:}
We added the physical explanation for super ballistic local range diffusion at L. 54. Interestingly, \citet{Lajeunesse2018} is also mathematically equivalent to \citet{Lisle1998}. They suggest two only two regimes are necessary, but they also state ``Here, we neglect this mechanism [sediment burial] and restrict our analysis to steady and uniform sediment transport.'' We have included sediment burial and derived the third stage proposed by Nikora, in addition to the two derived by Lisle et al. and Lajeunesse et al. 

\item 
Ln. 305. It isn't clear how the model confines the valid range for these types of studies. Please provide an example or additional explanation.

\paragraph{Reply:}
We added an additional sentence to clarify this point. 
\end{enumerate}
We sincerely appreciate your time reviewer \#2. Your comments greatly improved the manuscript.


\bibliography{biblio}
\end{document}