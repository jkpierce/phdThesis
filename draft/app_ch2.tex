%%!TEX root = diss.tex

\chapter{Calculations involved in dynamical sediment flux model}
\label{sec:appendixA}
\section{Derivation of the master equation for particle position}
\label{sec:appAmaster}
Here I derive the master equation \ref{eq:flippymaster} for the probability distribution of particle position from the Langevin equation \ref{eq:flippylangevin}. This is closely based off of the approach in \citet{Balakrishnan1993}.

The master equation satisfies $P(x,t) = \bra\bra \delta(x-x(t))\ket_\xi \ket_\eta$, where the averages are over all realizations of the two independent noises. It is most convenient to take these averages after taking Fourier transforms and manipulating time derivatives of the resulting equations.

Integrating the Langevin equation \ref{eq:flippylangevin}, substituting the solution in the above definition, and taking the Fourier transform in space gives
\be \tilde{P}(g,t) = \Big\bra  \Big\bra \exp \Big[- i g \int_0^t du [V+\sqrt{2D}\xi(u)]\eta(u) \Big]\Big\ket_\eta \Big\ket_\xi\ee
This can be rearranged as
\be \tilde{P}(g,t) = \Big\bra \exp\Big[i g V \int_0^t du \eta(u)\Big] \Big\bra \exp\Big[ i g \sqrt{2D}\int_0^t du \xi(u) \eta(u)]\Big]\Big\ket_\xi \Big\ket_\eta .\ee

Using the classic identity for an average over white noise of an exponential \citep{Balakrishnan1993,VanKampen2007} gives, after recognizing that the dichotomous noise satisfies $\eta^2 = \eta$ since $\eta$ is either $0$ or $1$,
\be \tilde{P}(g,t) = \Big\bra \exp\Big[(igV-g^2D) \int_0^t du \eta(u) \Big]\Big\ket_\eta. \label{eq:logi}\ee

Now I will take time derivatives in order to apply the so-called Furutsu-Novikov formula to evaluate the ensemble average over dichotomous noise \citep{Shapiro1978}. Applied to the dichotomous noise in figure \ref{fig:lislefig}, the Furutsu-Novikov formula is, for an arbitrary functional $F$ of the noise,
\be \pt \bra \eta(t)F[\eta(t)] \ket = \bra \eta \pt F \ket + k \big[\bra \eta \ket \bra F \ket -\bra \eta F \ket \big]  \label{eq:furuti}.\ee


Making the shorthands $G = igV-g^2D$ and $F=\exp G\int_0^t \eta(u)du$ and taking as time derivative of eq. \ref{eq:logi} gives
\be G^{-1}\pt \tilde{P}(g,t) = \bra \eta F\ket. \label{eq:furut}\ee
Taking a second time derivative opens up the possibility of using the Furutsu-Novikov formula \ref{eq:furuti}: 
\be G^{-1}\pt \tilde{P}(g,t) = \pt \bra \eta F\ket = G\bra \eta F \ket + k\big[ \bra \eta \ket \tilde{P}(g,t) - G^{-1} \pt \tilde{P}(g,t) \big]. \ee
Applying eq. \ref{eq:furut} finally gives
\be \pt^2 \tilde{P}(g,t)  = (igV-g^2D-k)\pt  \tilde{P} + k_E (igV-g^2D) \tilde{P},\ee
and inverse Fourier transforming provides the master equation \ref{eq:flippymaster}.

\section{Solution for the position probability distribution}
\label{sec:fluccymastersol}
The position probability distribution can be obtained from eq. \ref{eq:flippymaster} for the initial conditions $P(x,0) = \delta(x)$ and $ \pt P(x,0) = \frac{k_E}{k}\big[D\delta''(x)-V \delta'(x) \big]$ using Fourier transforms in space and Laplace transforms in time. Taking these transforms provides
\be \bar{\tilde{P}}(g,s) = \frac{s + k +\varphi  Dg^2 - i g V \varphi}{s(s+k)+(Dg^2-igV)(s+k_E)}, \ee
where $\varphi = k_D/k$ is the probability a particle starts at rest.

The numerator terms encode the initial conditions. The denominator terms are where the real structure of the solution is contained.
The numerator terms involving $g$ are easily expressed as first and second derivatives with respect to $x$. The problem, then is to calculate the inverse Fourier integral of the denominator.
This can be conducted by finding the zeros of the denominator, expanding in partial fractions, then applying the contour integral \citep[e.g.][]{Arfken1985}
\be \int_{-\infty}^\infty \frac{1}{2\pi i} \frac{e^{-i g x}}{g + i c} dg = \theta(x)e^{-c x}. \label{eq:contour}\ee

Rearranging the governing equation gives
\be \tilde{\hat{P}}(g,s) = \frac{\varphi D g^2 - ig V\varphi  + s + k}{D(s+k_E)}\frac{1}{g^2-i\frac{V}{D}g + \frac{s(s+k)}{D(s+k_E)}}\ee
The roots of the denominator are at
\be g_\pm = \frac{iV}{2D}\Big[ 1  \pm R \Big],\ee
where
\be R = \sqrt{1 + \frac{4D}{V^2}\frac{s(s+k)}{s+k_E}}\ee
The partial fractions expansion is therefore
\be \frac{1}{g^2-i\frac{V}{D}g + \frac{s(s+k)}{D(s+k_E)}} = \frac{D}{i V R}\Big[ \frac{1}{g-g_+} - \frac{1}{g-g_-}\Big]\ee
Giving 
\be \tilde{P}(x,s) = \frac{-\varphi D\px^2 + V\varphi\px + s + k}{VR(s+k_E)}\Bigg[ \int \frac{dg}{2\pi i}\frac{e^{-igx}}{g-g_+} - \int \frac{dg}{2\pi i}\frac{e^{-igx}}{g-g_-}\Bigg]\ee
or eventually
\be \tilde{P}(x,s) = \frac{-\varphi D\px^2 + V\varphi \px + s + k}{VR(s+k_E)}\Bigg[ \theta(x) \exp\Big(\frac{Vx}{2D}(1-R)\Big) + \theta(-x)\exp\Big( \frac{Vx}{2D}(1+R)\Big) \Bigg].\ee
This simplifies slightly to
\be \tilde{P}(x,s) = \frac{-\varphi D\px^2 + V\varphi \px + s + k}{VR(s+k_E)}\exp\Big[\frac{Vx}{2D} - \frac{V|x|}{2D}R \Big], \label{eq:laplaceapp}\ee
where as a reminder, $\varphi = k_D/k$.

Taking the inverse Laplace transform symbolically provides the desired probability density function
\be P(x,t) = \mathcal{L}^{-1}\Bigg\{ \frac{1}{V} \big[-\varphi D\px^2 + V\varphi \px + k + s \big] \exp\Big(\frac{Vx}{2D}\Big) \frac{\exp\Big(-\frac{V|x|R}{2D} \Big)}{(s+k_E)R} \Bigg\}(t). \ee
Using the shift property of Laplace transforms \citep[e.g.][]{Arfken1985}, this becomes
\be P(x,t) = \frac{1}{V}e^{-k_E t} \mathcal{L}^{-1}\Bigg\{ \big[-\varphi D\px^2 + V \varphi \px + k_D + s\big] \exp\Big(\frac{Vx}{2D}\Big) \frac{\exp\Big(\frac{-V|x|R_\ast}{2D}\Big)}{s R_\star} \Bigg\},\ee
where 
\be R_\ast = \sqrt{1 + \frac{4D(k_D-k_E)}{V^2}+ \frac{4D}{V^2}\Big(s-\frac{k_Ek_D}{s}\Big)}. \ee
Using the property that $\mathcal{L}^{-1}\big\{s \tilde{f}\big\} = (\delta(t) + \pt)f(t)$ \citep{Arfken1985} gives
\be P(x,t) = \frac{1}{V} e^{-k_E t} \big[ -\varphi D\px^2 + V\varphi \px + k_D + \delta(t) + \pt \big] \exp\Big[\frac{Vx}{2D}\Big] \mathcal{L}^{-1}\Bigg\{ \frac{-\exp\Big[\frac{-V|x|R_\ast}{2D}\Big]}{sR_\ast} \Bigg\}(t) \ee
Finally, using the identity
\be \mathcal{L}^{-1} \Big\{ \frac{1}{s} \tilde{g}(s-a/s)\Big\} = \int_0^t \mathcal{I}_0\Big(2\sqrt{au(t-u)}\Big) g(u) du
\ee
from \citet{Bateman1953}, pg. 133 (which can be derived by either u-substitution or series expansion in powers of $a$) provides:
\begin{multline}  P(x,t) = \frac{1}{V} e^{-k_E t}  \big[ -\varphi D\px^2 + V\varphi \px + k_D + \delta(t) + \pt \big] \exp\Big[\frac{Vx}{2D}\Big]\\ \times \int_0^t du \mathcal{I}_0\Big( 2 \sqrt{k_Ek_D u(t-u)}\Big) \\ \times \mathcal{L}^{-1}\Bigg\{ \frac{\exp\Big[ \frac{-V|x|}{2D}\sqrt{a+b s}\Big]}{\sqrt{a + b s}}\Bigg\}(u)  \label{eq:magic}\end{multline}
where $a = 1 + 4D(k_E-k_E)/(V^2)$ and $b = 4D/V^2$ are shorthands.

Using the trick of introducing a parameter and integrating over it to remove the square root from the denominator, the remaining transform can be evaluated from standard tables \citep{Prudnikov1992a}, eventually giving (after integration over and setting to 1 of the fake parameter),
\begin{multline} P(x,t) = \big[-\varphi D\px^2 + V\varphi \px + k + \delta(t)+ \pt \big]\int_0^t \mathcal{I}_0\Big( 2 \sqrt{k_Ek_D u(t-u)}\Big) e^{-k_E(t-u)} \\ \times \sqrt{ \frac{1}{4\pi D u}} \exp\Big[-k_D u - \frac{(x-Vu)^2}{4Du}\Big] du, \end{multline}
as reported in equation \ref{eq:monster}.


\section{Calculation of the scale-dependent rate function $\Lambda(T)$}
\label{sec:fluxconstant}
The central object required to calculate the sediment flux probability distribution in equation \ref{eq:flippyflux} is 
\be \Lambda(T) = \rho \int_0^\infty dx_i \int_0^\infty dx P(x+x_i,T).\ee
This represents the rate at which particles starts at $x_i$, anywhere to the left of $x=0$, and manage to cross $x=0$ by time $T$.

Again it is easiest to conduct the necessary calculus after Laplace transforming. This gives 
\be \tilde{\Lambda}(s) = \rho \int_0^\infty dx_i \int_0^\infty dx \tilde{P}(x+x_i,s).\ee
The Laplace transform of the rate function therefore follows after integrating equation \ref{eq:laplaceapp} twice.

Noting that $x+x_i$ is always positive,
The first integration gives
\begin{multline} \tilde{\Lambda}(s) = \rho \int_0^\infty dx_i \exp\Big[\frac{V(1-R)x_i}{2D}\Big]\\ 
	\times\Big(\varphi \frac{1-R}{2R(s+k_E)} - \frac{\varphi}{R(s+k_E)} - \frac{2D(s+k)}{V^2R(1-R)(s+k_E)}\Big), \end{multline}
and the second integration provides
\begin{multline}  \tilde{\Lambda}(s) = -\frac{\rho \varphi D}{V R(s+k_E)}  + \frac{2\rho D\varphi}{VR(1-R)(s+k_E)} \\ + \frac{4\rho D^2(s+k)}{V^3R(1-R)^2(s+k_E)}. 
	\label{eq:laplacefluxrate}\end{multline}

Taking the inverse transform, converting the $s$ factor in the numerator of the last term to $\pt + \delta(t)$, and using the shift property gives \citep[e.g.][]{Arfken1985}
\be \Lambda(t) = \rho \mathcal{L}^{-1}\Bigg\{ - \frac{\varphi D}{VR_\star s} + \frac{2D\varphi}{VR_\ast(1-R_\star)s} + \frac{4D^2\big( \cev{\pt} + k\big)}{V^3 R_\star(1-R_\star)^2s}\Bigg\}.\ee
Here the notation $\cev{\pt}$ means the derivative acts from the left on all terms multiplying it (as in $f(t)\cev{\pt} g(t) = \pt [f(t)g(t)]$).

After some work, using eq. \ref{eq:magic} and tabulated Laplace transform pairs \citep[e.g.][]{Arfken1985,Prudnikov1992a} to perform the inverse Laplace transforms, the net result for the rate constant is
\begin{multline} 
	\Lambda(t) = \rho \int_0^t \mathcal{I}_0\Big(2\sqrt{k_Ek_Du(t-u)}\Big)e^{-k_E(t-u)-k_D u} \\
	\times \Bigg[\sqrt{\frac{D}{\pi u}}\Big([\cev{\partial_t} + k]u-\frac{k_D}{2 k}\Big)e^{-V^2 u/4D}  \\ + \frac{V}{2}\Big([\cev{\partial_t} + k]u -\frac{k_D}{k}\Big) \erfc\Bigg(-\sqrt{\frac{V^2 u}{4D}}\Bigg)\Bigg] du,
\end{multline}
as reported in equation \ref{eq:fluxy}.

\label{sec:fluxlimits}
Equation \ref{eq:fluxy} is difficult to work with. To explore the behavior at extreme values of the observation time $T$, one can apply Tauberian theorems \citep[e.g.][]{Weiss1994} to invert the Laplace-transformed rate function of equation \ref{eq:laplacefluxrate} at the opposite extreme of $s$.
	
For example, at short times, expanding equation \ref{eq:laplacefluxrate} as $s\rightarrow \infty$ gives
\be \tilde{\Lambda}(s) = \frac{\rho k_E V}{2 k s^2} + \frac{\rho k_E}{k}\sqrt{\frac{D}{4 s^3}} \ee
which inverts to
\be \Lambda(t) \sim \frac{\rho k_E V T}{2 k} + \frac{\rho k_E}{k}\sqrt{\frac{D T}{\pi}},\ee
giving the small $T$ behavior.
This has two scaling limits within it. Provided that $T \ll 4D/V^2/\pi < 2 D/V^2$, the scaling goes as $\Lambda(T) \sim T^{-1/2}$. But if $T\gg 4D/V^2/\pi$, it goes as $\Lambda(t) \sim T$.

For large times, taking $s\rightarrow 0 $ gives
\be \tilde{\Lambda}(s) = \frac{\rho k_E V }{ks^2}, \ee
and this inverts to $\Lambda(T) = \rho k_E V T/k$ as the long time solution, providing a mean flux equivalent to the Einstein model (sec. \ref{sec:lisle}).
These limits are summarized in equation \ref{eq:fluccolimitcases}.
