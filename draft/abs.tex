%% The following is a directive for TeXShop to indicate the main file
%%!TEX root = diss.tex

\chapter{Abstract}

Bedload transport is the movement of coarse grains through river channels by bouncing, rolling, and sliding. Because coarse grains control river stability, predicting the rate of bedload transport is a fundamental problem in river science.
This problem is usually approached with continuum mechanics, but this approach is questionable considering that coarse sediment grains rarely move in densities approximating a continuum.

An alternative approach describes bedload transport from the trajectories of individual grains using statistical physics.
This approach has become increasingly popular in recent decades, but many fundamental issues prevent this approach from being widely adopted.
In particular, the connection between individual particle trajectories and transport rates remains unclear, and particle trajectory models remain highly simplified.
Feedbacks between topography and sediment transport remain challenging to analyze, and basic properties of bedload motions like downstream travel velocities remain incompletely understood.
Buried particles cannot move downstream, but even this simple observation has not been comprehensively described in the statistical physics approach.

This thesis presents four projects completed in my PhD which overcome these issues to provide new understanding of bedload transport from a statistical point of view.
First, I demonstrate how to calculate the sediment flux from the dynamics of individual grains, and I model the trajectories of grains alternating through motion and rest having fluctuating velocities in the motion state.
This work links the sediment flux to the grain-scale dynamics and describes particle trajectories in unprecedented detail.
Second, I include feedbacks between local bed elevations and sediment transport, quantifying the interplay of bed elevation changes and sediment transport rates, and predicting how long particles can stay buried in the riverbed.
Third, I incorporate this sediment burial process into a model of downstream sediment transport, predicting how grains move downstream when they can become buried.
Finally, I concentrate on bedload dynamics at short timescales, predicting the movement velocities of bedload particles using methods adopted from granular physics.
I conclude by summarizing these developments, discussing their implications for the statistical description of bedload transport, and suggesting how we can use this modelling progress to better understand landscapes.


\endinput


This research provides solutions to longstanding problems within the statistical approach to sediment transport. 



and the mathematical tools applied in this thesis should be of interest to researchers of many other problems. Most
This research has broad implications across river science where there is increasing acknowledgement of the shortcomings of the continuum hypothesis for predicting dynamics in mountain streams. This work extends a wide body of work on individual particle motions which originates from the 1930s, and unifies it with more recent approaches to calculate the bulk bedload flux which have not, until now, been related to the movement characteristics of individual grains.

A central task within Earth science is to understand and predict the evolution of landscapes due to flowing water.
As rainfall channelizes and flows downhill, it carves out basins and etches in networks of intersecting channels.
In the highlands of these networks, water arranges boulders and gravels into an intricate array of patterns -- steps, pools, bars, and riffles, which cooperate with vegetation both live and dead to set the stage for much of Earth's biota. In the lowlands, channels laden with old mountain sediments, weathered and abraded to fine particles, splay across floodplains and drift between configurations through millennia.

Sediment transport is a main driver of all fluvial dynamics.
Channel evolution ultimately occurs because individual sediment grains move from one location to another. Yet in a majority of modelling studies, landscapes are represented as continua, where the locations of individual grains are averaged away, and sediment transport is represented as a steady stream of mass, rather than the intermittent movements of individual grains
Useful as this continuum approach may be, many fluvial phenomena are not well-suited for it.
Sediment transport rates are known to fluctuate widely through space and time, and these fluctuations are understood to initiate bedform development and control channel widths. The largest grains in small mountain channels are known to confer the most stability, while violating any assumption of being small compared to the scales of interest. To understand fluvial dynamics, we need the capability to model discrete grains, not just continua.

In this thesis, I present four or five years of my theoretical research into the movements of individual sediment grains in river channels.
This sub-field of Earth science, with a focus on grain-scale process, has existed at least since Hans Albert Einstein in 1937, and within it, modellers have traditionally compromised on severe approximations to balance realism against mathematical difficulty.
In this enterprise, perfectly flat beds which do not change shape, spherical grains, infinite movement velocities, and turbulence free flows have been typical assumptions to make progress, even though they have little basis in reality.
The research presented here takes on more mathematical difficulty than before to introduce more realism, adding some bricks to the fortress, and introducing new methods to the field which should work well for other people to make more bricks later. I hope you find it useful! Happy reading.
