\section{Scope: identifying future research directions}
\label{sec:scope}







\subsection{recap} 
Einstein developed first stochastic theories describing the diffusion and flux of bedload. 
Ancey 2006, building on earlier work from Lisle 1998 and the many investigators who criticized and revised Einstein, formalized Einstein's assumptions over a foundation in Markov process theory, and extended his work by treating the bedload flux within a control volume as the cooperation of many independent two-state Markov processes: each bed particle within the control volume makes random transitions between motion and rest. 
Thus he derived a statistical distribution of the bedload rate which reproduces Einstein's result when the mean is taken. 
However, the magnitude of fluctuations predicted by this model are too small. 

Ancey 2008 extended the Einstein theory by linking the transition rates between particles within the observation window with a collective entrainment term: they prescribed that the transition rate from rest to motion depends on the number of active particles. 
This is a simple way to include the effects of coherent turbulence and impact-based entrainment: it is an experimental observation that moving particles tend to come in waves \citep{Drake1988}. 
The Ancey 2008 model was demonstrated capable of describing fluctuations, and all of its parameters have physical meanings, but notably there is no clear suggestion as to how collective and individual entrainment processes can be separated within experiments. 
Clearly, approaches to derive entrainment probabilities are focused on the individual entrainment rate \citep{Dey2018}, and there have been no developments with regard to computing the collective entrainment rate: there is no clear physical model of collective entrainment yet. 

Turowski 2009 extended the Ancey work to take account of limited supply, in order to describe semi-alluvial channels where alluvial deposits lie on top of bedrock. 
He continued to work within the Einstein paradigm by describing the transport rate within a control volume, rather than by counting the number of particles leaving the control volume: although the Ancey 2008 work essentially outlines this possibiilty. The transport rate at a point should be considered as the number of emigration events in a unit time. 

Heyman et al 2013 and Ma et al 2014 went on to consider the statistics of emigration: they tried to discern what could be learned about the transport rate at a fixed point from the control volume formalism of Ancey et al and Einstein before that. 
Heyman et al was concerned with the statistics of the waiting time between successive emigration events. 
The original Einstein (1937, 1950) assumptions generate an exponential waiting time ditsribution with a timescale related to the entrainment rate.  
However, Heyman et al (2013) showed that when collective entrainment effects were included there are two timescales: one fast timescale related to the individual entrainment rate, and one slower timescale related to collective entrainment effects. 
Thus, the distribution of waiting times between successive emigration events, which is a useful concept for alternative stochastic models of the bedload flux distribution \citep[e.g.][]{Turowski2010}, is a more subtle object if collective entrainment occurs. 
These effects are accentuated at low transport rates, where the fast and slow timescales are more disparate. 
At high rates, the timescales become comparable and the waiting time between successive emigration events blends into an Einstein-like exponential. 

All of these approaches considered bedload transport within a control volume or at a point, but indeed, a very large set of contemporary river science studies are concerned with the diffusion or spreading of bed material through a downstream reach of river \citep{Hassan2016}. 
Bedload diffusion is not captured by a model at a single location: often, it has been described using an advection diffusion equation. 
These advection diffusion equations follow from considerations of mass balance within river channels, but they were not derived from any underlying microstructural model until \citep{Ancey2014}. 
\citet{Ancey2014} extended the previous control volume model \citep{Ancey2008} to an array of control volumes, where emigration from the $i$th volume is immigration to the $i+1$th volume. 

In an approach widely used in chemical physics \citep{Gardiner1983} and ecology \citep{Bailey1968}, \citet{Ancey2015, Ancey2015} derived an advection diffusion equation from their microstructural model in a limit as the size of each control volume goes to zero while the number becomes infinite. 
Thus, in their interpretation, bedload diffusion emerges in the continuum limit of coupled birth-death immigration-emigration models. 
They rederived the famous Exner equation of bedload diffusion. 

Taken together, this research exhibits a coherent progression from the original work of Einstein to a more mathematically and physically based set of more general approaches. 
Within these approaches the bedload flux is understood as a random quantity. 
The mean bedload flux and the magnitude of its fluctuations are modelled in an unambiguous way \citep{Ancey2006, Ancey2008} considering no limitations in sediment availability, and it was shown that collective effects are a necessary inclusion to properly describe fluctuations in bedload transport \citep{Ancey2008}. 
An extension to a more realistic situation of finite supply was developed \citep{Turowski2009}, and the relationships between these control volume based models and the definition of bedload flux as particles crossing a plane perpendicular to the flow were developed \citep{Heyman2013, Ma2014b, Ballio2014}. 

These approaches did not address spatial correlations or variations in bedload fluxes, although the problem of bedload diffusing through a reach of channel is of contemporary significance \citep{Hassan2016}. 
This extension was the most recent development in birth-death modelling of bedload \citep{Ancey2014, Ancey2015}. 
By extending the \citet{Ancey2008} model to an array of adjacent control volumes, an advection-diffusion equation describing the concentration of bedload in motion was developed, providing the first connection between a microscale stochastic model of bedload transport with a macroscopic advection-diffusion formulation. 
\citet{Heyman2014} examined the spatial correlations between local bedload fluxes expressed by the \citet{Ancey2014} model. 
This is the state of the art of bedload flux models. 

\subsection{Assumptions, calibraiton problems, and unclear aspects: A criticism of the birth-death approach} 


First, the various glaring assumptions of the birth-death approaches are highlighted. 
\begin{enumerate}
\item Models only consider one particle size, while natural streams exhibit wide distributions of sizes and bedload transport expresses a wide set of effects related to particle size segregation \citep{Wilcock2003, Parker1982, Chen2008}. 
\item There is no clear means to discriminate collective and individual entrainment processes within experiments. 
\item There is only a one-way feedback in these models: bedload is considered subordinate to the fluid flow, and the effect of bedload transport back onto the fluid flow is considered negligible. In fact, there are measurable influences of bedload transport on the fluid phase \citep{}; it's not yet clear whether a one-way coupled scheme such as this can actually describe natural streams across a realistic range of conditions. 
\item Collective entrainment is somewhat of a catch-all term with the physical mechanisms which may contribute to it unresolved. These may include collective entrianment, the formation and disintegration of particle clusters, interactions between the motion of grains of idfferent size, turbulent structure, local avalanche behavior, and bed form migration. 
\item there have been no theories developed to compute the collective entrainment rate from any simplified mechanical model, although there is a very large set of work concerned with calculating the individual entrainment rate from considerations of fluid turbulence and random granular arrangement \citep{Einstein1949, Einstein1950, Grass1970, Paintal1971, Cheng1998, Wu2004, Dey2008, Tregnaghi2012, Dey2018}. Calculating the collective entrainment rate from underlying principles appears on the surface very difficult. If it is considered to stem from collisions of moving grains with stationary grains, the collective entrainment probability will be related to the probability of collisions with the granular bed. If it is considered to stem from granular avalanches, where coherent turbulent structures initiate collections or clusters of grains into motion simultaneously, the collective entrainment probability will depend on the collective dynamics of the granular assembly -- leading immediately into the murky physics of force balance within granular assemblies.
\item Einstein-like assume a clean division between rest and motion states, which is no doubt an idealization. Bedload transport makes a continuous transition from the idealized start-stop motions of Einstein-like models to a less idealized granular flow or creep, where all particles move together in a coorelated fluid-like flow \citep{}, meaning bedload models should only hold at relatively low mobility stages. 
\item At the same time, the divison of motion into only two categories may be flawed. A wide collection of studies have highlighted different modes of motion. \citet{Einstein1950} considered that bedload was particles moving "in a rolling, sliding, or saltating mode". What if rolling, sliding, and saltating modes of motion were treated independently? In an \citet{Ancey2006} model of independent particles switching between states, this would require a four-state model. There are too many transition probabilities within such a model probably to calibrate the model from experiments. Again, in order to consider such a scenario we would need more knowledge of underlying mechanics. 

\end{enumerate}

Incorporating multiple grain sizes in birth-death models is possible, in principle. 
As a first approximation, it is easy to introduce multiple grain sizes without including interactions between grain sizes in the entrainment and deposition probabilities for each size fraction. 
However, spatial and temporal heterogeniety in the bed surface characteristics are a hallmark of bedload transport of gravel mixtures \citep{Hassan2008}. 
This leaves models such as \citet{Ancey2008} somewhat without basis when multiple grain sizes are considered. 
The concept of \citet{Turowski} is essential to include when multiple grain sizes are considered: the bed surface state determines which grains can entrain \citep[e.g.][]{Wilcock2003, Parker1982}, and presumably which grains can deposit, as well. 
The entrainment and deposition rates of each size fraction will need to depend on the bed surface state, and fractional transport will set up spatial hetereogenieties in the bed surface state, so that an \citet{Ancey2014} type model, incorporating the possiblity of spatial hetereogeniety, should be mixed with a \citet{Turowski2009} type model to incorporate the effect of the bed surface state on the entrainment and deposition rates of each size fraction. 
This extension will not be easy, and it will introduce many undetermined parameters: to pin down this transport model of multiple interacting size fractions will take a lot of work. 


These stochastic models are admittedly difficult to calibrate. 
Their calibration requires a large dataset which is prohibitive to measure within natural streams. 
Therefore, their applicability in real streams remains limited until the inputs of stochastic models can be computed from physical theories based upon practically measurable quantities. 
This quest for relationships to compute the inputs of stochastic models from measurable quantities has been called "stochastic closure" for the analogous problem in turbulence \citep{Heyman2016}. 
The difficult issues precluding the application of stochastic models to natural streams should not be downplayed: much more work is needed. 



\begin{comment} 
Birth death processes are a special type of stochastic process. 
Stochastic processes track the evolution of a set of random variables through time. 
Birth death processes consider a discrete state space, such as the size of a population. 
Members of the population are born and die: transitions are allowed between adjacent sites in the discrete state space. 

Within sediment transport, Einstein was the first to consider a model for the bedload flux in terms of stochastic processes. 
He considered the transport of individual bedload grains to be a random switching between motion and rest states. 
In this way, supplementing this consideration with empirical reasoning, Einstein developed a formula for the mean transport rate \citep{Einstein1950}. 
Einstein determined that the bedload transport rate should follow a Poisson distribution. 

Einstein's predecessors have extended his formulations in order to bring them in accord with experimental data and to phrase them in terms of a stronger mathematical foundation drawn stochastic process theory. 
These extensions probably began with \citet{Lisle1998}. 
They linked Einstein's theory of tracer dispersion \citep{Einstein1937}, which is contingent on random transitions between states of motion and rest, into Markov process theory, showing that Einstein's tracer dispersion theory was just a two-state Markov process where the states are motion and rest.  
Then \citet{Ancey2006} went on to rephrase the \citet{Einstein1950} bedload flux derivation in terms of a two-state Markov process, where random transitions between states of motion and rest are included into a Chapman-Kolmogorov equation -- a general mathematical formulation. 
In this way, they derived a probability distribution for a bedload flux which is based upon underlying randomness in the entrainment and deposition of bedload grains. 
Taking the mean of their bedload flux probability distribution reproduces an Einstein-like formula for the mean bedload flux. 
This work is sketched to indicate how Markov process theory can work to describe the bedload flux. 


\begin{enumerate}
\item all einstein-like models only hold where a clean division between motion and rest states exists. this will break down at high transport rates when bedload becomes a sheet flow with continuous particle motion 
\item 
\item physical and mechanisistic basis of collective entrainment is foggy
\item models are limited to one grain size 
\item these models only have one way feedback between grains and fluid
\item models are difficult to calibrate
\item models are difficult to actually test -- as they get more flexible they can fit anything. they contain ad hoc parameters which must be adjusted. so far they have only been fitted on highly idealized uniform or semi-uniform sediment grains 
\item all in all, extension to natural scope is precluded
\item bedforms induce spatial differences in the model parameters 
\item "as fluctuations span scales larger than the ones at which bedload can be considered stationary and homogeneous, oor even at scales larger than the experiment size, average equations fail at describing the nonlinear .. " 
\end{enumerate} 

\section{Summary: scope for extension} 

From the basis provided in this literature I'll now make a few small extended birth-death models. 

1. diffusion in 2d (horizontal and vertical)
2. availability of depositional sites on the bed surface
3. Extension to multiple grains sizes: A formal outline 



\end{comment} 

