
\subsection{The bedload flux at a point: the counting statistics of emigration events} 

Up to this point, the Markovian approaches to understand a fluctuating bedload rate were based upon analyzing the stochastic number of active particles within a control volume. 
The bedload flux definition considered for these studies was somewhat unconventional. 
Typically it is defined as the rate of bedload crossing a vertical plane perpendicular to the flow direction \citep{Ballio2014}. 

The studies of \citet{Heyman2013} and \citet{Ma2014b} demonstrated that the birth-death model of \citet{Ancey2008} can also be used to define a flux at a position in space. 
Although the birth-death model of \citet{Ancey2008} counts the number of moving particles at each instant of time, it does not count the number of emigration events which contribute to this number. 
Counting the number of emigration events in an interval of time is a much more difficult problem.
Ma et al (2014) approached this counting problem using a path integral approach: this approach is essentially outlined in \citet{Ohkubo2009}. 
The path integral approach draws heavily from the stochastic simulation algorithm of \citet{Gillespie1977}. 

The conventional definition of the bedload flux is an integral of particles through a surface perpendicular to the flow: 
\be Q_s(t) = \int \int_A dA \textbf{u}_s \cdot \textbf{n}, \ee
where $\textbf{n}$ is a unit vector perpendicular to the average flow direction and $\textbf{u}_s$ is the velocity of bed load. 
This instantaneous flux is very difficult to measure, so usually it is averaged over some time interval $\delta t$: 
\be Q_s(t;\delta t) = \frac{1}{\delta t} \int \int_A \int_t^{t+\delta t} \textbf{u}_s \cdot \textbf{n} d\tau dA = \frac{\delta V_s}{\delta t} = v_s \frac{\delta S(t;\delta t)}{\delta t}. \ee
$Q_s(t;\delta t)$ is the volume flux averaged over $\delta t$; $\delta V_s$ is the volume of bed load particles passing through $A$ within $\delta t$; $S(t)$ is the cumulative number of particles passing through the cross section since t=0, and $\delta S(t;dt)$ is the number of particles passing through the surface between $t$ and $t+\delta t$, i.e. $\delta S(t; \delta t) = S(t+\delta t) - S(t)$. 
This equation means the bedload flux at $t$, when viewed in a time-averaged sense, is the number of particles crossing the surface between $t$ and $t+\delta t$. 
Because volume units can always be multiplied in later, the bedload flux is written 
\be q_s(t) = \frac{\delta S(t; \delta t}{\delta t}. \ee

This bedload flux is a random variable.
A fluctuation is defined strictly as $q_s' = q_s - \bra q_s \ket$, and the magnitude of fluctuations is defined without ambiguity by the variance $\bra q_s'^2 \ket$. 
This $\delta S(t;dt)/\delta t$ is the number of emigration events between $t$ and $t+\delta t$, so if this number could be counted, the bedload flux would be known at the downstream boundary of the control volume: the bedload flux would be known at a point in space. 

The path integral formulation performs this counting problem. 
For the sake of computation, the time interval t is discretized as $i \Delta t = t$, $i=0,1,2,\dots$.
In the end, the limit $\Delta t \rightarrow 0$ will be taken. 
If the increments $\Delta t$ are sufficiently small, Ma et al argued that emigration events during one of these intervals will be approximately Poissonian so that
\be P[\Delta S_i = n | N_i(t) = N_i] = \exp[-f_3(N_i)\Delta t] \frac{(f_3(N_i) \Delta t)^N}{n!}. \ee
This equation expresses the probability that $n$ particles emigrate between $t$ and $t+\delta t$ is a Poisson distribution with rate parameter $f_3(N_i)\Delta t$, where $f_3(N_i) = \gamma N_i$ is the emigration rate. 

Now Ma et al argued that because the entrainment rate $f_1 = \lambda + \mu N \~ O(L)$, the deposition rate $f_2 = \sigma N \~ O(L)$, and the emigration rate $f_3 = \gamma N \~ O(1)$, if the observation window is sufficiently large, entrainment and deposition processes dominate, so that the number of moving particles $N$ is perturbed only weakly by emigration. 
In this limit of sufficiently large observation window size, it then holds that
\be P(\sum \Delta S_i = n | N_1,N_2, \dots) = \exp[-\sum f_3(N_i)\Delta t]\frac{(f_3(N_i)\Delta t)^n}{n!}, \ee
which in the continuum limit becomes a path integral counting emigration events: 
\be P[\delta S = n |N(t): t_0<t<t_0+\delta t] = \exp[-\int f_3(N(\tau))d\tau]\frac{[\int f_3[N(\tau)]d\tau ]^n}{n!}.\ee
Denoting the integral over the process $f_3(N(\tau))$ as $\alpha = \int_{t_0}^{t_0+\delta t} d\tau f_3[N(\tau)]$, the probability for $n$ emigration events in $\delta t$ is written 
\be P(\delta S = n) = \int d\alpha \frac{e^{-\alpha}\alpha^n}{n!}P(\alpha,t). \ee
From this difficult derivation a simple formula emerges: the probability of $n$ emigration events in $\delta t$ is an integral over all possible realizations of the process $N(t)$, where each process is given a weight $e^{-\alpha}\alpha^n/n! P(\alpha,t) d\alpha$. This is called a Cox process. It's a generalized Poisson process where the parameter $\alpha$ is also a random variable. It represents a superposition of a series of Poisson distributions taking every possible path of $N(t)$ into account. 
This is a stochastic description of the bedload transport flux based on the counting statistics of emigration events. 

Notably, because of the time integral of $N(t)$ involved in this relationship, it has a memory. It is non-Markovian. 
This character reveals a very interesting fact: the fluctuations of the transport rate depend on the time scale of observation. 
In fact, Ma et al derived the expression for the variance: 
\be \delta t \text{var} [q_s(\delta t)] = \gamma^2 \frac{\lambda (\sigma + \gamma)}{(\sigma + \gamma -\mu)^2} 2 t_c [\delta t - t_c (1-e^{-\delta t/t_c}] + \frac{\gamma\lambda}{\sigma + \gamma -\mu}\delta t \label{eq:3regimes} \ee

where $t_c = 1/(\sigma + \gamma - \mu)$ is the autocorrelation time of $N(t)$, representing memory in the system. 
Apparently, the variance of the bedload flux at a point depends on the observation or averaging time-scale $\delta t$. 

In fact, Ma et al discriminated three regimes of behavior from \ref{eq:3regimes}:   
\be
    \delta t \text{var}[q_s(\delta t)] \approx 
\begin{cases}
    \frac{\gamma \lambda}{\sigma + \gamma - \mu}, & 0<\delta t \ll t_l \\
    \delta t \frac{\gamma^2 \lambda(\sigma+\gamma)}{(\sigma+\gamma-\mu)^2}   & t_l<\delta t< t_c \\
    2 \gamma^2 \frac{\lambda(\sigma+\gamma)}{(\sigma+\gamma-\mu)^2}t_c + \frac{\gamma \lambda}{\sigma + \gamma - \mu}   & t_c \ll \delta t < \infty
\end{cases}
\ee
When this three-regime equation is divided by the mean bedload rate $\text{mean}[q_s(\delta t)]$, the behavior is discriminated by a dimensionless number Ma et al called $Ra$: 
\be Ra(\delta t) \approx \begin{cases} 1 & 0<delta t \ll t_l \\
\frac{delta t}{t_l} & t_l < \delta t < t_c \\ 
\frac{2 t_c}{t_l} + 1 & t_c \ll \delta t < \infty \end{cases} \ee

The essential point is that Ma et al derived three regimes of fluctuations from the \citet{Ancey2008} model. 
Depending on the sampling interval $\delta t$, three different scaling relations are possible for the magnitude of fluctuations in the bedload flux. 
These intervals are discriminated by two timescales $t_c$ and $t_l$. 
There are probably deep consequences of this research which remain to be fully understood. 
At this stage, it's sufficient to note that fluctuations differ with observation timescale, and that this effect can be discriminated into stages with a dimensionless number $Ra$. 

The origin of the two timescales still needs to be discussed. 

There is another way to get information about the bedload flux at the downstream boundary of the control volume defined by \citet{Ancey2008}, and this was examined by \citet{Heyman2013, Heyman2016}. 
Rather than actually counting the number of emigration events with the complicated path integral approach, Heyman et al. took a somewhat simpler angle: they examined the waiting time between successive emigration events. 

