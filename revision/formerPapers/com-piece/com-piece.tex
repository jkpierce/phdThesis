\documentclass{article}
% General document formatting
\usepackage[margin=0.7in]{geometry}
\usepackage[parfill]{parskip}
\usepackage[utf8]{inputenc}
\usepackage{subfig}         % side-by-side figures 
% Related to math
\usepackage{amsmath,amssymb,amsfonts,amsthm}
\usepackage{graphicx}
\usepackage{natbib}
%\bibliographystyle{unsrtnat}



% \usepackage[math]{kurier}
\usepackage{setspace}       % \onehalfspacing and \singlespacing
\newcommand\be{\begin{equation}} % shortcut to start eq envs 
\newcommand\ee{\end{equation}}   % shortcut to end eq envs
\newcommand\ol{\overline}        % shortcut to draw overline 
\newcommand\bra{\langle}
\newcommand\ket{\rangle}
\usepackage[utf8]{inputenc}
\DeclareUnicodeCharacter{2010}{-}% support older LaTeX versions
\doublespacing

\begin{document}

\title{Does gravel flow through rivers, or does it bounce?\\ the complementary ideas of Bagnold and Einstein}
\author{Kevin Pierce}
\maketitle

\begin{abstract}
\noindent Coarse sediment transported as bedload within a turbulent water stream is a mechanical process, which motivated two main approaches to understand it based on physical principles: the stochastic approach of Einstein and the deterministic approach of Bagnold. 
These two approaches have become prevalent research paradigms which support a conceptual divide in the river science community. 
By reviewing the Einstein and Bagnold paradigms and cross-comparing them, I argue this divide is short-sighted. 
Calling on wider knowledge on the dynamics of gravel bed rivers, I speculate about the role of these mechanistic concepts of bedload transport within future geophysical theories of gravel river morphology.  
\end{abstract}


\section{Mechanistic theories of bedload transport}


Gravel-bed rivers are troubled natural systems, and changes in their morphology are linked to a wide array of ecological problems \citep{Newson2000, Gaeuman2017}.  
Since the transport of coarse material as bedload is an important control on channel morphology \citep{Church2006, Recking2016}, bedload transport has been carefully researched for over a century. 
Despite this effort, our ability to predict bedload transport rates remains poor.
Predictions of bedload transport can deviate from measured values by factors of 100 or 1000 \citep{Gomez1989, Barry2004, Recking2012}, motivating an intense pursuit of deeper understanding. 
 
Bedload grains move in traction with the bed by rolling, sliding, and bouncing \citep{Einstein1950, Bagnold1973}. 
Since this transport is governed by Newtonian mechanics, it should be fully characterized by exchanges of momentum between the granular and fluid phases.
This expectation highlights a mechanical description of bedload transport as an attractive prospect.
However, bedload transport results from a turbulent flow interacting with granular matter, so a full mechanical description would be a fusion of two notoriously difficult and tenuously understood subfields of classical physics, meaning many of the required details are missing.
In absence of these details, many researchers have attempted to construct mechanical descriptions using supplementary hypotheses. 
From over a century of efforts, two have emerged as especially popular research paradigms. 
These are the stochastic approach, initiated by \citet{Einstein1937, Einstein1950, Einstein1964} and the deterministic approach, initiated by \citet{Bagnold1956, Bagnold1966, Bagnold1973}. 

These two paradigms are founded on the application of mechanical reasoning to simplified concepts of bedload motion.
These concepts involve empirical parameters, so these models must be calibrated to applications. 
Viewing bedload transport phenomena in terms of these simplified concepts, proponents of the Einstein and Bagnold paradigms take differing views on most aspects of the process, which often leads to disagreements and miscommunications between proponents of each school. 
This conflict coincides with the observation that neither mechanical approach is, practically speaking, very useful.
Both paradigms generate poorer bedload transport predictions than simple power law curves, calibrated to empirical data without reference to mechanical concepts \citep{Barry2004}, and because of their semi-empirical attributes, the superiority of either paradigm across the full range of bedload conditions is difficult to judge \citep{Iverson2013}. 

Accordingly, although each mechanical approach has merits, the perspective that either is superior is not supported by evidence. 
Meanwhile, there is no clear way to join the two paradigms together, so proponents of each toil away in their respective camps, solving problems within (or with) their chosen paradigm, and failing to communicate science across the conceptual rift between approaches. 
We need to ask: is this rift justified by evidence? 
Are the stochastic and deterministic approaches to understand bedload transport mechanically really so different that they share no common ground, and is one or the other completely wrong? 
To address these questions, we should analyze the concepts, successes, and short-comings of each approach, and use these analyses to form a cross comparison.  
To break anticipation, the short answer is no.  


\section{The Einstein paradigm: mechanistic-stochastic}


Einstein introduced stochastic concepts to understand the bedload transport rate by relating it to individual bedload motions \citep{Einstein1942, Einstein1950}.
He conceptualized the individual motion as a series of steps with intervening rests, and he set out to determine the resting time and step length with flume experiments \citep{Einstein1937}. 
During these experiments, he realized that these motion characteristics were not consistent or predictable from one particle to the next, so he was forced to switch to a statistical conception.
Across a population of many particles, Einstein realized the resting time and step length of individual particles generated statistical distributions, which he obtained from his experiments \citep{Einstein1937}. 

From this statistical concept of individual bedload motions, Einstein made two key developments. 
First, he calculated the probability distribution of the effective velocity of bedload grains, as they move downstream through cycles of steps and rests. 
From the velocity difference between the fastest and slowest grains, he derived the rate of spreading or diffusion of tracer particles \citep{Einstein1937}.
Second, he derived the random character of the resting period as a mechanical consequence of fluid turbulence, and used this to calculate the mean rate of bedload transport \citep{Einstein1950}. 
The tenacity with which Einstein's stochastic concepts of bedload motion have been modified, critiqued, and improved over the last seventy years is a testament to the originality of his ideas \citep{Ettema2004}. 

Generations of researchers have worked within the paradigm established by Einstein, criticizing and improving his concepts without fully invalidating them.  
Characteristics of individual motions have been investigated extensively \citep{Crickmore1962, Hubbell1964, Yano1969, Nakagawa1976, Drake1988, Hassan1991, Habersack2001, Ancey2008, Roseberry2012, Heyman2013, Fathel2015, Heyman2016}, and Einstein's statistical perspective appears essentially correct. 
Since Einstein, other motion statistics been measured and incorporated into stochastic theories of bedload transport, including the velocity, acceleration, and travel time during single particle motions \citep{Drake1988, Radice2006, Ancey2008, Lajeunesse2010, Furbish2012a, Roseberry2012, Fathel2015, Furbish2016,  Heyman2016}. 
Bedload diffusion has been carefully analyzed, and the problem is obviously more nuanced than Einstein considered. 
Diffusion has different characteristics depending on the temporal or spatial scales of consideration \citep{Nikora2002, Zhang2012, Martin2012}, and this probably relates to the effects of particle burial \citep{Yang1971, Nakagawa1980, Voepel2013, Martin2014, Bradley2017} and variable movement characteristics from sediment size \citep{Fan2017} on effective velocity differences between moving bedload. 

Einstein's mechanical theory of the mean bedload transport rate \citep{Einstein1950} has been more carefully phrased in relation to underlying mechanics, and his concepts have been extended to compute full statistical distributions of the bedload rate \citep{Sun2000, Ancey2006, Ancey2008}. 
Subsequent theories on the resting time of stationary sediment, which is equivalent to the probability of erosion of a grain \citep{Yalin1972}, took more careful account of the mechanics driving particles into motion, including variable supporting configurations of bed surface grains, and the forcing from turbulent lift and drag \citep{Paintal1971, Wu2004, Dey2018}.
Meanwhile, contemporary experiments have increased recognition of the combined role of the magnitude and duration of turbulent fluctuations on the erosion process \citep{Diplas2008, Valyrakis2010, Celik2014}, which Einstein-like models of the resting time (or erosion rate) have only begun to incorporate \citep{Tregnaghi2012,Dey2018}. 

Instead of computing mean bedload rates only, contemporary theories have leveraged stochastic mathematics \citep{Cox1965} to compute probability distributions of the bedload flux, deriving mean values and the expected magnitude of bedload fluctuations \citep{Sun2000, Ancey2006, Ancey2008}. 
These descriptions are a more complete description of bedload transport \citep{Ancey2008}, and the higher-order statistics provide additional benchmarks against which models can be tested \citep{Iverson2013}. 
This research links back to the resting time of stationary bedload in a contentious way which requires resolution. 
Einstein's concepts, when translated to theories which predicts the magnitude of bedload fluctuations, generate fluctuations which are too small to match experiments \citep{Ancey2006}. 
Accordingly, some contemporary models have introduced feedbacks in particle motion, which could be physically attributed to the erosion of multiple grains from the bed at once, and these somewhat ad hoc prescriptions imply realistically large fluctuations \citep{Ancey2008, Heyman2013, Ma2014}, meaning their mechanical origin deserves clarification. 

From this sketch of research within the Einstein paradigm, we can see that Einstein's initial concepts have generated a great deal of new understanding of the physics underlying bedload transport, but they have also generated many open questions.
The Einstein paradigm is particularly successful at describing bedload fluxes at low transport stages, where instantaneous fluctuations in bedload rates are appreciable relative to mean values, taking values up to 400\% as large \citep{Bohm2004, Ancey2008, Singh2009, Heyman2016, Gonzalez2017}.
Under these conditions, Einstein's stochastic concept, where bedload transport is dominated by the random motions of individual grains, must be closer to the underlying physics. 
 
However, Einstein-type theories have key limitations.
First, at high transport rates, the assumptions of these theories are not realistic. 
In these conditions, the entire bed surface moves as a type of granular flow, with motion several particle diameters below the surface \citep{Jenkins1998, Mouilleron2009, Frey2014, Houssais2015}, so Einstein's partition of transport into distinct periods of motion and rest breaks down \citep{Ancey2008, Heyman2014}.
Second, when channel morphology affects bedload motions, for instance by size sorting of bed material \citep{Nelson2010, Sun2015}, the influence of bedforms \citep{Iseya1987, Cudden2003, Kasprak2014, Hassan2017}, or aggradational/degradational cycles \citep{Dhont2018} there is limited understanding of how this can be represented in the statistics characterizing individual motions. 
Accordingly, we conclude that Einstein-type models are close to the underlying physics of low bedload discharges over a nearly flat bed \citep{Ancey2008, Heyman2013, Ma2014}, but at high bedload discharges, or when morphology affects bedload transport, new developments are required. 


\section{The Bagnold paradigm: mechanistic-deterministic} 


In contrast, Bagnold understood the bedload rate as a deterministic energy balance between the solid and fluid phases. 
His key concept was that gravity supplies excess energy to the fluid flow, beyond the capacity of turbulence to dissipate, so it is dissipated by the friction of moving bedload against the stationary bed \citep{Bagnold1956, Bagnold1966, Bagnold1973}.
To complete this concept, Bagnold had to define what constituted excess energy within the fluid flow. 
To this end, he made a now-famous hypothesis: when bedload transport is at equilibrium value, so it dissipates all excess power available to move sediment, the average shear stress on the bed will match the threshold of sediment motion.
This is called the Bagnold hypothesis \citep{Seminara2002, Ancey2006}. 
Supplementing his energy balance concept with this hypothesis, Bagnold predicted a bedload rate ($q_b$) which scales, at high stages of transport, as the average fluid shear stress ($\tau$) to the $3/2$ power: $q_b \propto \tau^{3/2}.$
Bagnold's theory requires the specification two parameters: one characterizing the dynamic friction of moving bedload, and another characterizing the threshold of bedload motion \citep{Bagnold1973}. 

Bagnold's ideas are enlightening for their pragmatic clarity, and they have been taken up and improved by many researchers. 
His deterministic concept of bedload motion is simplified just enough from the real complexity to make contact with the reliable mechanical principle of conservation of energy, and at high bedload rates, his predicted scaling between bedload rate and average shear stress ($q_b \propto \tau^{3/2}$) is in accord with classic empirical results \citep{MeyerPeter1948}, and, strikingly, is also in accord with the most advanced numerical simulations of bedload transport \citep{Schmeeckle2014}.
However, despite these successes, his simplified assumptions have left behind many open questions for proponents of his ideas. 

For example, Bagnold's theory of bedload transport ignores the details of bedload trajectories, which has motivated many efforts to reconcile his average energy budget concepts with the dynamics of individual motions \citep{VanRijn1984, Bridge1984, Wiberg1989, Bridge1992, Nino1998, Bialik2015}. 
Since bedload trajectories are a consequence of turbulent forcing and bed collisions, whose characteristics are random, this work is a juxtaposition of fluctuating and average energy concepts. 
A shortcoming of this research is that the details of forces imparted by a turbulent flow on bedload grains are incompletely understood \citep{Schmeeckle2007, Dwivedi2010, Dwivedi2011}, especially when particles are not spherical \citep{Maxey1983a}, with relatively important terms entering the force balance apart from the typical drag and lift \citep{Bialik2015}.
This may partly explain why these Bagnold-type models, based on the Newtonian dynamics of individual bedload motions, remain troubled. 
They usually overpredict bedload rates, and underpredict the energy dissipation of bedload transport \citep{Bridge1984, Bialik2015}. 

Some authors have attributed these shortcomings to a lack of inclusion of 
 collisions between moving grains \citep{Lee2002}, and to a lack of understanding of the stresses imparted by moving grains on the stationary bed \citep{Nino1998}.
Both of these criticisms really point toward a deeper problem which links deeply to granular physics.
With its energy budget perspective on bedload motion, which ignores the details of individual motions, Bagnold's original theory essentially phrases bedload motion as an average granular flow with a highly simplified internal friction law, or rheology \citep{Frey2014}. 
Within granular physics, tractable models of rheology are limited to two extremes: there is the simple quasi-static rheology assumed by Bagnold (Coulomb-type), which should be valid in high concentration flows \citep{Berker1992}, and there are models adapted from the kinetic theory of gases \citep{Jenkins1998}, valid in low concentration flows. 
Possibly, bedload is an intermediate concentration flow, which means its description as a type of granular flow demands a theory of rheology which does not exist \citep{Frey2011, Frey2014}. 
There are mixed conclusions from experiments on the validity of Bagnold's friction law \citep{Silbert2001, Frey2014, Houssais2016}, and in context of the shortcomings of trajectory based bedload models, this topic deserves further attention.  

From this short review of Bagnold's concepts and the research directions they have generated, it's easy to conclude that Bagnold's work pushed science forward. 
At high bedload discharges, his description appears essentially correct: when bedload transport resembles a collective flow, it scales with average shear stress as $\tau^{3/2}$.
In this regime, his energy dissipation arguments must be close to the relevant physics of bedload transport. 
However, Bagnold's theory yields poor results at low transport stages and over arbitrarily sloping beds, where it fails to characterize bedload transport \citep{Engelund1976, Luque1976, Seminara2002, Martin2000}.
One issue is that measured bedload rates express large fluctuations at low mean discharges, so mean values poorly represent the bedload signal at low rates, and statistical representations are really closer to the underlying physics \citep{Ancey2008}. 

The deepest issue with Bagnold at low transport stages is that solid transport rates are not suitably large to dissipate the excess energy available to move bedload grains, meaning Bagold-type models do not fit experimental data without calibrating bulk particle friction coefficients to unphysical values \citep{Engelund1976, Luque1976, Nelson1995, Nino1998}.
For example, \citet{Nino1998} found an effective friction angle of $56.6$ degrees was required to describe their experiments at low transport stage, a value well above the angle of repose of any natural sediment \citep{Miller1966}.
We can conclude that Bagnold's ideas are close to the physics of bedload transport at high transport stages, but his theory fails outside of the conditions in which bedload moves as a collective flow. 
Accordingly, for lower bedload discharges, we need a different perspective. 


\section{Stochastic or Deterministic?}


Clearly, both mechanistic paradigms have strengths and weaknesses, but is one or the other relatively closer to the physics of bedload transport?  
Often, the rift between the Einstein and Bagnold paradigms is framed in terms of a divide between stochastic and deterministic approaches, and this is short-sighted. 
Scientists are not justified in ignoring consistently reproduced findings: in natural streams and controlled laboratory measurements, bedload transport rate time-series always appear with random fluctuations. 
In laboratory channels with steady uniform flows and nearly identical or narrowly graded sediment, under conditions where the development of channel morphology is suppressed, instantaneous bedload rates can be up to 400\% mean values \citep{Bohm2004, Ancey2008, Roseberry2012, Heyman2016}.
In real gravel bed rivers and in flume experiments where channel morphology can evolve, bedload fluctuations may be even more predominant \citep{Iseya1987, Kuhnle1988, Hoey1992, Cudden2003, Singh2009, Dhont2018}. 
Experiments consistently show us that bedload transport rates are statistical quantities.

Accordingly, a full characterization of bedload transport should include mean bedload rates, as well as any statistics of the bedload time-series, such as the expected magnitude of instantaneous deviations from mean values. 
In general, bedload transport would be more completely described by a stochastic theory. 
However, this does not mean Einstein is better than Bagnold, or that one or the other is completely wrong.
I have already discussed that Einstein appears closer to the physics at low bedload rates, when the intermittency of individual motions dominates the bedload signal, while Bagnold appears closer at high bedload rates, when bedload motion appears like a collective flow. 
Experimental data reinforce this assignment of the idealized frameworks of Einstein and Bagnold to complementary extremes of transport phenomena. 

For example, \citet{Ancey2008} measured the bedload transport rate ($q_s$) of identical glass beads within a narrow flume under a range of steady flow conditions. 
They found the relative strength of bedload fluctuations, characterized by the ratio of the bedload variance to the mean ($\text{var}(q_s)/\overline{q}_s$), scaled with average fluid shear stress. 
This indicates a key trend of bedload transport, at least when bedform development is suppressed: stochastiscity is accentuated at low mean bedload rates, and it is suppressed at high bedload rates. 
They went further to analyze their results in relation to Bagnold's assumption that the energy dissipation of bedload governs its rate. 
From their data, they estimated the fractions of energy dissipated by turbulence and by bedload transport. 
At low bedload discharges, as much as $90$ percent of the energy supplied by gravity was disspated by turbulence (so $10$ percent by bedload), while at high bedload discharges, a much as $75$ percent of  flow energy was dissipated by momentum transfers with moving bedload (so $25$ percent by turbulence).
These trends suggest that bedload transport is actually a compromise between the stochastic concept of Einstein -- that bedload transport is dominated by individual motions, and the deterministic concept of Bagnold at high discharges -- that bedload transport is governed by an excess energy budget.
At low bedload discharges, the data favor Einstein's concepts; while at high bedload discharges, they favor Bagnold's.  
 
It's best for science if the opposition between the stochastic and deterministic viewpoints dissolves, and if river scientists start to see Einstein and Bagnold as what they are: idealized models which pertain to particular limits of the full phenomena of bedload transport. 
Proponents of Einstein build bedload transport rates up from individual particle motions, and contend that the motion characteristics of individual particles are not predictable due to the random character of fluid turbulence and the arrangement of bed grains, and this is correct. 
When bedload discharges are low, so that motions are sparse, the random character of individual motions is emphasized within the bedload rate, and it appears as a statistical quantity.
Under these conditions, a full statistical description based on individual motions is probably more realistic \citep{Ancey2008, Heyman2013, Ma2014, Heyman2016}.
In contrast, proponents of Bagnold understand bedload transport as a dissipative flow, where the details of individual motions are less relevant than their collective dynamics. 
When bedload discharges are high, the random character of individual motions is washed out through inter-granular collisions and mixed through the population of moving grains, so average characteristics dominate the bedload rate, and a deterministic description based upon collective properties of a granular flow is probably more descriptive \citep{Jenkins1998,  Frey2011, Frey2014}.

The physics of bedload transport are apparently complex enough to support more than one theoretical description, each pertinent to a different regime where particular interactions are emphasized, which is not surprising. 
After all, should we be surprised to find the coupled physics of turbulent flows and granular media to be simpler than either alone?
Bagnold and Einstein, or deterministic and stochastic -- these are both meaningful physical descriptions, but of different conditions of bedload transport.
This leads to a few final questions. 
Is one condition or the other of bedload transport more relevant to understand gravel bed rivers physically, and how exactly does bedload transport link into a more complete geophysical understanding of gravel bed rivers?  


\section{Toward deeper understanding of gravel bed rivers} 


Our understanding of gravel bed rivers, of which bedload transport is only one component, remains far too primitive to supercede descriptive geomorphology.   
\citet{Schumm1977}, in his classic work, outlined morphology as the form and structure of river channels which results from coordination between the supply, storage, and transport of sediment.
In gravel bed rivers, which may flow through forests, or previously glaciated landscapes, sediment supply and storage may be more deeply linked to geology \citep{Rennie2018}, ecology \citep{Wohl2017}, climate \citep{Gregory2006, Slaymaker2009}, and human influence \citep{Hooke2000, Gregory2006, Slaymaker2009}, than they are to bedload transport: so channel morphology could be largely imposed by controls external to river channels.
Yet channel form cannot evolve or develop without the erosion and redistribution of bed sediment, so in gravel bed rivers, and in coordination with these external controls, bedload transport underlies channel morphology \citep{Church2006, Recking2016}. 

Gravel bed rivers are facing serious ecological pressures \citep{Frissell1993, Hauer2016}, and this undoubtedly links to changes in their morphology, driven by external controls, while the scope of these pressures pivots on our capacity to understand and counter these morphological changes \citep{Wohl2015}.
Accordingly, we can identify two inquiries which are essential to a deeper understanding of gravel bed rivers.
First, research into the external controls on channel form, like geological and forest processes, needs to continue. 
Storage and supply are just as important as transport in determining channel form. 
Second, we need a quantitative understanding of the erosion and redistribution of bed sediment with channels, and this is where Einstein and Bagnold come in.
However, we must accept the geophysical problem is really more difficult than either scientist envisioned. 

Within gravel bed rivers, depending on the particular conditions of sediment supply and storage, a wide diversity of organized structures can be discerned on the bed \citep{Church1998, Hassan2008, Nelson2014, Venditti2017}.  
These morphological elements span a range of spatial scales: clusters, ribs, cells, steps, bars -- from the scale of several grains to several times the width of channels. 
These bedforms modify the stability of bed grains, and redirect hydraulics, in turn coordinating bedload transport \citep{Laronne1976, Lisle1992, Kasprak2014, Recking2016, Hassan2017}.
Simultaneously, bedload transport can form, move, and destory bedforms \citep{Whiting1988, Hoey1992, Cudden2003}, and morphological units such as pools can support quasi-periodic cycles of aggradation and degradation \citep{Jackson1982, Dhont2018}.  
From these observations, we can conclude that real gravel bed rivers express feedbacks between transport and morphology which are far beyond the scope of our simplified models, and we can speculate that these feedbacks are a predominant cause of the noted failure of bedload models when they're applied to natural channels \citep{Gomez1989, Barry2004, Recking2012}. 

Within natural rivers, transport of gravels has been divided into three stages to classify its intensity \cite{Jackson1982, Ashworth1989}. 
Stage 1 is finer material overpassing a locally static bed, so any mobile material originates upstream. 
Stage 2 is partial mobility of local bed material, so at any one time, part of the bed is static, but any grain might eventually move. 
Stage 3 is full mobility of bed material, so no part of the bed is static, and all grains move equally, in what could be called a flow. 
We can associate stages 2 and 3 with the complementary concepts of Einstein and Bagnold, if we acknowledge they overlook many details of natural rivers, such as transport-morphology feedbacks. 

If we agree with \citet{Wolman1960} that, owing to their relative frequency of occurence, intermediate transport stages predominate over high stages in terms of geomorphic influence, then Einstein's concepts are somewhat more relevant to the geophysics of gravel bed rivers. 
Bedload moves intermittently under most conditions of geomorphic interest. 
However, the Einstein paradigm's inability to understand the interaction of transport with channel morphology, such as the collective migration of bedforms, suggests maybe there's something to learn about this from the energy balance concept of Bagnold. In the classic literature on bedform movement, and its role in bedload fluctuations, these motions are described as a collective migration of very many grains at once \citep{Whiting1988, Hoey1992, Cudden2003}, and is this really so different from Bagnold's concept of bedload motion? 

A possible conjecture is that in gravel bed rivers, individual grains and bedforms may both have stochastic motions, and while individual grains move in accord with turbulence, might bedforms move as a type of collective flow? 
Can variable driving and resisting forces govern individual movements of bedload, while an energy balance governs the collective movements of bedforms? 
Or is the mix of phenomena even more intricate? 
One imagines the alternate start stop motions of a bedload sheet, which is not more complex than cycles of aggradation and degradation in pools. 
How could the fluid and grains coordinate to express these phenomena? 
In building a geophysics of gravel bed rivers, we might need to get comfortable with complexity, and to attempt a descriptive synthesis from the concepts of simpler models, like the bedload theories of Einstein and Bagnold.  

\bibliographystyle{agu}
\bibliography{biblio}
\end{document}

