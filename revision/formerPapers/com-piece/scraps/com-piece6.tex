\documentclass{article}
% General document formatting
\usepackage[margin=0.7in]{geometry}
\usepackage[parfill]{parskip}
\usepackage[utf8]{inputenc}
\usepackage{subfig}         % side-by-side figures 
% Related to math
\usepackage{amsmath,amssymb,amsfonts,amsthm}
\usepackage{graphicx}
\usepackage{natbib}
\bibliographystyle{unsrtnat}


% \usepackage[math]{kurier}
\usepackage{setspace}       % \onehalfspacing and \singlespacing
\newcommand\be{\begin{equation}} % shortcut to start eq envs 
\newcommand\ee{\end{equation}}   % shortcut to end eq envs
\newcommand\ol{\overline}        % shortcut to draw overline 
\newcommand\bra{\langle}
\newcommand\ket{\rangle}

\begin{document}

\title{Does gravel flow in rivers, or does it bounce?\\ similarity and difference between Bagnold and Einstein}
\author{Kevin Pierce}
\maketitle

\begin{abstract}
\end{abstract}
\section{Mechanistic theories of bedload transport}

Gravel-bed rivers are troubled natural systems, and changes in their morphology are linked to a wide array of ecological problems \citep{Gaeuman2016}.
Since the transport of coarse material as bedload is an important control on channel morphology \citep{Church2006, Recking2016}, bedload transport has been the subject of intense research for over a century. 
Despite this effort, our ability to predict bedload transport rates remains poor, and predictions of bedload transport can deviate from measured values by factors of 100 or 1000 \citep{Gomez1989, Barry2004, Recking2012}. 
Bedload grains move intermittently by rolling, sliding, and bouncing along the river bed \citep{Einstein1950, Bagnold1973}. 
Since this transport is governed by Newtonian mechanics, it should be fully characterized by exchanges of momentum between the granular and fluid phases, highlighting a mechanical description as an attractive prospect.
However, bedload transport results from a turbulent flow interacting with granular matter, so a full mechanical description would be a fusion of two notoriously difficult and tenuously understood subfields of classical physics, meaning many of the required details are missing.
In absence of these details, many researchers have attempted to construct mechanical theories of bedload transport using hypotheses where necessary, and from over a century of effort, two have emerged as important research paradigms. 
These are the stochastic approach, initiated by \citet{Einstein1937, Einstein1950, Einstein1964} and the deterministic approach, initiated by \citet{Bagnold1956, Bagnold1966, Bagnold1973}. 

These two paradigms are founded on the application of mechanical reasoning to simplified concepts of bedload motion.
These concepts involve empirical parameters, so these mechanistic models must be calibrated to applications. 
Owing to these simplified concepts, proponents of Einstein and Bagnold-type models take different perspectives on most aspects of bedload transport, which often leads to disagreements and miscommunications between proponents of each school. 
This conflict coincides with the observation that neither mechanical approach is, practically speaking, very good.
Both paradigms generate poorer bedload transport predictions than simple power law curves, calibrated to empirical data without reference to mechanical concepts \citep{Barry2004}
And the superiority of either paradigm across the full range of bedload conditions is difficult to judge, owing to their empirical parameters \citep{Iverson2013}. 
Accordingly, although each mechanical approach has merits, the perspective that either is generally superior, for any reason apart from practicality, is not supported by evidence. 
Meanwhile, there is no clear way to join the two paradigms together, so proponents of each toil away in their respective camps, solving problems within (or with) their chosen paradigm's simplified concepts, in context of a general failure to communicate science across the conceptual rift. 
We need to ask: is this rift justified by evidence? 
Are stochastic and deterministic approaches to understand bedload transport as a mechanical process really so different that they share no common ground, and is one or the other completely wrong? 
This commentary explores this question. 


\section{The Einstein paradigm: mechanistic-stochastic}

% einstein concepts
% einstein successes
% - his concept of indivdual motions as random is essentially correct
% - his research provoked a deeper study and understanding of diffusion and the statistics of particle motion
% - stochastic bedload transport, when coupled to fluid flow, shows a linear instability which implies bedform generation Jerolmack Ancey Bohorquez
% einstein shortcomings
% - we expect bedload to dissipate energy but this is not included in formulations
% - diffusion has been difficult and more complicated than expected 
% - einstein fails at high bedload rates because motion-rest states break down and we have a granular sheet flow
% einstein takeaway 
% - more work is needed, but it... 
% - succeeds at low transport rates, and 
% - stochastic transport generates bedforms 

% Einstein concepts
Einstein initiated the stochastic paradigm by relating bedload rates to individual particle motions \citep{Einstein1942, Einstein1950}.
He conceptualized the transport of an individual particle as an alternating series of rests and hops, and he set out to determine the period of rest and distance of one hop with flume experiments \citep{Einstein1937}. 
During these experiments, he realized that the rest period and hop distance of individual particles was not consistent or predictable from one particle to the next, so he was forced to switch to a statistical perspective. 
Viewing motion characteristics over a population of many particles, Einstein realized the resting period and hop distance of individual particles generated statistical distributions, and he measured these distributions in his experiments \citep{Einstein1937}. 
Initially, he used these distributions to calculate the probability distribution of the effective velocity of bedload grains as they move downstream through cycles of rest and motion. 
Because there is an effective velocity difference between the fastest and slowest grains, he derived the rate of spreading or diffusion of tracer particles as they move downstream \citep{Einstein1937}. 
Later, he derived the random character of the resting period as a mechanical consequence of fluid turbulence, and he used this concept to derive the mean rate of bedload transport \citep{Einstein1950}.

% einstein successes
% - his concept of indivdual motions as random is essentially correct
% - his research provoked a deeper study and understanding of diffusion and the statistics of particle motion
% - stochastic bedload transport, when coupled to fluid flow, shows a linear instability which implies bedform generation Jerolmack Ancey Bohorquez
Einstein's stochastic approach to understand bedload transport has been criticised and improved by many researchers, without abandoning his essential ideas. 
Since Einstein, statistical characteristics of individual motions have been investigated extensively and incorporated into stochastic theories of bedload transport, including characteristics he did not originally consider, such as the velocity, acceleration, and travel time of particles \citep{Hubbell1964, Yano1969, Nakagawa1976, Hassan1991, Habersack2001, Ancey2008, Roseberry2012, Fathel2015, Heyman2016}.
Einstein's conception of bedload motion characteristics as random variables appears essentially correct.
Understanding of particle diffusion has advanced a great deal, and the problem is obviously more nuanced than Einstein considered.  
Particle diffusion has different characteristics depending on the temporal or spatial scales of consideration \citep{Nikora2002, Zhang2012, Martin2012}, and this probably relates to the effects of particle burial \citep{Sayre1971, Nakagawa1980, Voepel2013, Martin2014, Bradley2017} and variable movement characteristics from sediment size \citep{Fan2017} on the differences in effective velocities of moving bedload. 
A key success of Einstein's theory is that stochastic bedload transport, 



% einstein shortcomings
% - we expect bedload to dissipate energy but this is not included in formulations
% - diffusion has been difficult and more complicated than expected 
% - einstein fails at high bedload rates because motion-rest states break down and we have a granular sheet flow



% einstein takeaway 
% - more work is needed, but it... 
% - succeeds at low transport rates, and 
% - stochastic transport generates bedforms 


Generations of researchers have evaluated, criticised, and improved Einstein's stochastic concepts of bedload transport, without abandoning his basic ideas. 
Statistical characteristics of individual motions have been investigated extensively \citep{Hubbell1964, Yano1969, Nakagawa1976, Hassan1991, Habersack2001, Ancey2008, Roseberry2012, Heyman2013, Heyman2016}, and Einstein's conception of bedload motion characteristics as random variables appears essentially correct. 
Individual bedload grains move randomly, but in accord with statistical patterns dictated by turbulence and configuration of bed grains.  
Since Einstein, other characteristics of particle motion have been measured and incorporated into stochastic theories of bedload transport, such as the velocity, acceleration, and travel time during single particle motions \citep{Drake1988, Radice2006, Ancey2008, Lajeunesse2010, Furbish2012a, Roseberry2012, Furbish2015, Fathel2015, Heyman2016}. 


Einstein's mechanical theory of the bedload transport rate \citep{Einstein1950} has been reworked a great deal. 
Subsequent theories deriving the resting time of stationary sediment, which is equivalent to the rate of erosion of an individual surface grain \citep{Yalin1972}, took account of variation in the supporting configuration of bed surface grains, in addition to variable forcing from turbulent fluctuations in their Newtonian analyses \citep{Paintal1971, Wu2002, Dey2018}.
Meanwhile, contemporary experiments have increased recognition of the role of both the magnitude and duration, or the impulse, of turbulent fluctuations on particle entrainment \citep{Diplas2008, Valyrakis2010, Celik2014}, which Einstein-like models of the resting time (or erosion rate) have not yet incorporated \citep{Dey2018}. 
Instead of computing mean bedload rates only, contemporary theories have leveraged the mathematics of stochastic processes to compute probability distributions of the bedload flux, deriving mean values and the expected magnitude of bedload fluctuations \citep{Sun2000, Ancey2006, Ancey2008}. 
These descriptions provide a more complete description of bedload transport \citep{Ancey2008}, and higher order statistics of bedload transport provide additional benchmarks to test models against experiments \citep{Iverson2013}. 
This work has linked back on the resting period between particle movements in a contentious way which still requires resolution. 
Einstein's concepts, when translated to theories which predicts the magnitude of bedload fluctuations, generate fluctuations which are too small to describe experiments \citep{Ancey2006}. 
Accordingly, some contemporary models have incorporated collective feedback mechanisms in particle motion, which could be physically attributed to the erosion of multiple grains from the bed at once \citep{Ancey2008, Heyman2013, Ma2014}, as might arise from small granular avalanches \citep{Heyman2013}, the effects of coherent turbulence on particle motion \citep{Nino1998, Amir2014, Santos2014, Shih2017}, or the migration of bedforms \citep{Dhont2018}. 

Bedload theories within the Einstein paradigm have provided a great deal of understanding about the physics of bedload transport; and his conception is clearly more representative of bedload fluxes at low transport stages, where instantaneous fluctuations in bedload rates are appreciable relative to mean values, taking values up to 400\% as large \citep{Bohm2004, Ancey2008, Singh2009, Heyman2016}. 
However, Einstein type theories have several key limitations, especially at high transport stages and with spatial or temporal differences in transport characteristics.  
For one thing, Einstein-type theories are energetically flawed, because they make no reference to the energy carried by the flow; intutively, we expect that the water stream cannot hold an unlimited number of particles in motion, because these particles dissipate energy in their transportation, and the flow has a finite amount. 
Under high transport stages, Einstein's partition of bedload transport into separate periods of motion and rest breaks down, because the entire bed surface moves as a sheet flow, with motion several particle diameters below the surface \citep{Jenkins1998, Mouilleron2009, Houssais2015}. 
When transport characteristics differ in space or time, such as differences in the stability of bed sediment caused by morphology, or the spatial sorting of different size grains, researchers of the Einstein paradigm have very limited capacity to predict this difference in bedload transport statistics, meaning much more research is needed. 
Finally, no limiting factor of bedload rates within einstein-type models 
We can conclude although Einstein-type models are close to the underlying physics of low bedload discharges over a nearly flat bed \citep{Ancey2008, Heyman2013, Ma2014}, but at high sediment discharges or for any discharge over an irregular bed, a different description is required. 

\section{The Bagnold paradigm: mechanistic-deterministic} 
% bagnold concept
% bagnold successes
% - describes transport rate at high discharges Julien 1994
% - in accord with many classic experiments and contemporary numerical simulations 
% bagnold shortcomings
% - determinisitic formulation contrasts with key feature of bedload rates at low discharges -- fluctuations
% - bagnold fails at low discharges because friction coefficient must be fit to unphysical values 
% - bagnold's hypothesis is flawed, so his formulation gives poor results over arbitrarily sloping beds
% - if bedforms result from a loss of linear stability in coupled fluid-solid system, bagnold does not capture requisite physics Balmforth 2001 
% bagnold takeaway
% - bagnold is realistic at high bedload rates, but 
% - the failure at low rates is a consequence of the underlying assumption that bedload transport is always a dominant energy dissipation mechanism
% - however this contrasts with ancey findings, 
% - we conclude that bagnold's range of validity is limited to high bedload rates over relatively flat beds 

% bagnold concept
Bagnold formed a different simplified concept of bedload transport to understand it with mechanical principles, initatiating the deterministic paradigm of bedload theory. 
From a physical point of view, Bagnold considered bedload transport as a momentum exchange between solid and liquid phases.
He balanced the average power supplied to the flow by gravity against the average frictional dissipation by moving bedload, and neglected any details of turbulent fluctuations, irregularities in the configuration of stationary grains, or the trajectories of moving particles.
Bagnold's formulation hinges on the application of an average dynamic friction coefficient to moving bedload, in order to characterize its energy dissipation. 
His formulation is built upon the so-called Bagnold hypothesis \citep{Bagnold1973, Engelund1976, Luque1976, Seminara2002, Ancey2006}: when bedload transport is at its equilibrium value, so it dissipates all available power to move sediment, Bagnold held that the tractive force imparted by the moving fluid on the bed surface will match the threshold of sediment motion.
Bagnold's original formulation predicts a bedload transport rate which scales, at high stages of transport, as the average fluid shear stress to the $3/2$ power: $q_b \propto \tau^{3/2},$ which is in accord, at high bedload discharges, when morphology is poorly developed, with a large set of experiments and numerical simulations \citep{MeyerPeter1948,Gomez1989,Schmeeckle2014, Elghannay2017}.
For this success, his perspective on bedload transport has been taken up by generations of river scientists.
Bagnold's scientific legacy is even more interesting because his research seems to be at most a tertiary component of his life's work, with a much larger portion of his time devoted to his decorated career as a military commander in both world wars, and to his hobby of driving cars around the Sahara desert, which eventually earned him status as a preminent desert explorer \citep{Bagnold1988}. 

Bagnold's original concept is enlightening for its pragmatic approach \citep{Ashida1972, Bagnold1973, Engelund1976, Luque1976}. 
His deterministic concept of bedload motion is simplified just enough from the real complexity to make contact with the foundational and reliable mechanical principle of conservation of energy. 
His theory has been influential for a clear ability to describe bedload transport from mechanical concepts at relatively high transport stages, when the influence of morphology is minimized, and for its accord with classic results which also predict a scaling of bedload transport as the average shear stress to the $3/2$ power \citep{MeyerPeter1948}.
At the same time, his highly simplified assumptions have left many open questions for proponents of his ideas, and his key hypothesis has been undermined by theoretical arguments \citep{Seminara2002}. 

For example, Bagnold's theory of bedload transport ignores the details of bedload trajectories, which has motivated many efforts to reconcile his average energy budget concepts with the dynamics of individual bedload motions. 
The time spent in suspension by bedload grains between successive collisons, which is highly variable in accord with fluid turbulence and collisions with the bed surface \citep{Bialik2015}, contribute to the relative amount of time that bedload grains spend in contact with the bed, meaning the dynamic friction coefficient Bagnold introduced should scale with the the details of bedload trajectories, which are themselves contingent on turbulence. 
Accordingly, many investigators have tried to understand the turbulent basis of bedload trajectories, and to incorporate these details with Bagnold's average energy budget concept, which should fix the number of particles in motion and hence the bedload rate \citep{Abbott1977, Bridge1984, Wiberg1989, Bridge1992, Nino1998}.
This research is a juxtaposition of average and turbulent energy considerations, as it effectively contends that the number of particles in motion results from an average power balance, while the dissipation mechanism which limits the number of particles in motion is contingent on fluid turbulence and collisions between moving and stationary grains. 
A shortcoming of this research is that the details of forces imparted by a turbulent flow on bedload grains are incompletely understood \citep{Schmeeckle2007, Dwivedi2010, Dwivedi2011}, especially when particle shapes deviate from spherical \citep{Maxey1989}, with relatively important terms entering the force balance apart from the typical drag and lift terms \citep{Bialik2015}.
For this reason, and others, these Bagnold-type models based on the Newtonian dynamics of individual bedload motions remain troubled, usually overpredicting bedload rates \citep{Bridge1984, Bialik2015}. 
This problem has been attributed to a lack of inclusion of collisions between moving grains \citep{Lee2002}, and a lack of understanding of stresses imparted by moving grains on the stationary bed \citep{Nino1998}, so both of these topics deserve further research attention \citep{Bialik2015}. 

The theoretical argument which undermines Bagnold's hypothesis is based upon the addition of a relatively small transverse (cross-stream) tilt in the bed surface \citep{Seminara2002}. 
Under such conditions, \citet{Seminara2002} showed that no bedload transport rate, no matter how large, is sufficient to reduce the fluid shear stress at the bed to the critical value.
This undermines the Bagnold hypothesis, even for nearly horizontal beds, as the hypothesis is apparently not robust to even minor deviations from ideal conditions. 
\citet{Parker2003} ammended the Bagnold hypothesis by replacing it with an Einstein-type concept regarding the balance of erosion and deposition rates, although they used ad hoc relationships between erosion and deposition rates and the average fluid shear stress to form this connection. 
Nevertheless, these efforts key into a similar stream of research which borrows many concepts from Bagnold, balancing erosion and deposition rates parameterized by average characteristics of the fluid flow, to generate bedload transport predictions with the same scaling patterns as Bagnold's original formulas. Although these theories successfully describe experiments, they do not defer to Bagnold's average energy balance concept \citep{Charru2004, Charru2006, Lajenesse2010, Lajenesse2018}. 

The key failure of Bagnold-type theories is at low transport stages, where they fail to characterize bedload transport \citep{Engelund1976, Luque1976, Francis1977, Ancey2008}.
One issue is that measured bedload rates express relative large fluctuations at low mean discharges, so mean values poorly represent the bedload signal at low rates, and statistical representations are really closer to the underlying physics \citep{Ancey2008}.
However, the deepest issue with Bagnold at low transport stages is that solid transport rates are not suitably large to dissipate the excess energy available to move bedload grains, meaning Bagold-type models do not fit experimental data without calibrating bulk particle friction coefficients to unphysical values \citep{Engelund1976, Luque1976, Nelson1995, Nino1998}.
For example, \citet{Nino1998} found an effective friction angle of $56.6$ degrees was required to describe their experiments at low transport stage, a value well above the angle of repose of any natural sediment \citep{Miller1966}.
This failure of Bagnold-type theories at low rates is a consequence of the underlying assumption that bedload transport is always a dominant energy dissipation mechanism of the flow. 
\citet{Ancey2008} estimated energy dissipation by turbulence and by momentum transfers with moving bedload.
They found at low solid discharges, as much as $90$ percent of the energy supplied by gravity was disspated by turbulence (so $10$ percent by bedload), while at high solid discharges, a much as $75$ percent was dissipated by momentum transfers with moving bedload (so $25$ percent by turbulence). 
The data suggest that Bagnold's energy balance assumptions are close to the physical processes of bedload transport at higher discharges over a nearly flat bed, but at lower discharges, or when the bed is irregular, a different physical description is required. 

\section{Stochastic or Deterministic?}

% at low bedload discharges in the absence of morphology you have stochastic properties, while at high bedload discharges they become deterministic 
% there is an effective population averaging when many particles are in motion 
% accordingly bedload transport has different characteristics at low and high transport rates, and einstein and bagnold mirror these differences 
% reference ancey's energy dissipation findings
% reference ancey's findings on the scaling of bedload rates 

The rift between the Bagnold and Einstein research paradigms is not justified, because these two paradigms pertain to different idealized conditions of bedload transport: relatively high or low solid discharges, respectively, over flat or nearly flat beds.
At low discharges, bedload rates are highly variable, even under the most controlled laboratory conditions available. 
For example, in laboratory experiments at low bedload discharges under uniform steady flows, with identical glass beads or narrowly graded gravel as sediment, over flat or nearly flat beds, instantaneous bedload fluxes can take on values as much as 400\% mean values \citep{Bohm2004, Ancey2008, Heyman2016}. 
However, as mean bedload discharge is increased, the relative strength of bedload fluctuations decreases, so the mean bedload rate becomes more representative of bedload transport \citep{Ancey2008}. 
Proponents of Einstein build bedload transport rates up from individual particle motions, and contend that the motion characteristics of individual particles are not predictable due to the random character of fluid turbulence and bed grain configuration. 
When bedload discharges are low, so that motions are relatively rare, the random character of individual motions is emphasized within the bedload rate, and it appears as a statistical quantity, so a full statistical description based on individual motions is closer to the relevant physics \citep{Ancey2008, Heyman2013, Ma2014, Heyman2016}.
In contrast, proponents of Bagnold understand bedload transport from average characteristics of bedload motion. 
When bedload discharges are high, so that many individuals combine to form a continuous sheet flow, the random character of individual motions is washed out through inter-granular collisions, so average characteristics dominate the bedload rate, and a deterministic description based upon collective properties of a granular flow is really closer to the relevant physics \citep{Jenkins1998, Hsu2004, Mouilleron2009, Frey2011, Houssais2016, Maurin2018}.
Accordingly, an application or comparison of Bagnold and Einstein-type models across a wide range of transport stages is not meaningful, because bedload transport has different characteristics at low and high discharges. 
It's heavily stochastic at low discharges, and quasi-deterministic at high discharges. 


% so bagnold and einstein describe two extremes 

\section{Toward deeper understanding of gravel bed rivers} 

% wolman's frequency and magnitude argument suggests that intermediate bedload conditions can be a dominant control on morphology
% schumm implies this linkage is actually more complex, and involves the interplay of storage and supply as well 
% In light of these perspectives, we can assocaite particular conditions of storage and supply, in conjunction with intermediate bedload conditions, as the conditions which foster morphology creation
% morphology has feedbacks with sediment transport
% meaning intermediate bedload regimes support strong morphology-transport feedbacks


\citet{Wolman1960}, in their classic theoretical work on geomorphology, introduced a concept of balance between the frequency and magnitude of hydaulic conditions in reworking river channels.  
Although high bedload discharges, like those described by the Bagnold paradigm, may be incredibly geomorphically active, and they may drastically change river morphology over short timescales, these discharges are rare. 
The majority of geomorphic change may result from intermediate bedload discharges which are both relatively common and competent enough to overcome the stability of river channels and enact small changes in their morphology over longer intervals of time.  
\citet{Shumm1960}, in another classic work, introduced a complementary perspective of river morphology. 
He viewed it as an interplay between the storage, supply, and transport of sediment. 
Joining these perspectives, we can associate particular conditions of sediment storage and supply, along with intermediate conditions of bedload transport, between the extremes of Bagnold and Einstein, as the genesis of a wide diversity of morphological elements within gravel bed rivers \citep{Brayshaw1984, Church1998, Hassan2008, Nelson2014, Venditti2017}.
These morphological elements, including clusters, ribs, bars, and so on, provide stability to bed grains and modify in-channel hydraulics, in turn coordinating bedload transport \citep{Laronne1976, Lisle1992, Wilcock2003a, Kasprak2014, Recking2016, Hassan2017}. 
We can conclude in the intermediate bedload conditions which neither the Einstein or Bagnold paradigms capture, there is an intricate feedback between the formation, migration, and destruction of collective sedimentary forms.
Individual motions generate morphological structures, which tend to migrate or disintegrate in collective motion.  
The exotic phenomena of bedload transport, in the intermediate region between Bagnold and Einstein, are the feedbacks between transport and morphology.   

% the intermittency of individual motion, informed by turbulent fluctuations, develops morphological elements
% under particular storage / supply conditions, 
% these form, migrate, and disintegrate in an interplay between individual and collective transport events
% and this superposition of individual and collective motions imparts distinct signatures to bedload fluctuations across multiple temporal and spatial scales
% accordingly, the phenomenology of the intermediate bedload regime is really a fusion of Einstein and Bagnold
% Bagnold fails to generate bedforms, while Einstein succeeds, meaning a stochastic component of transport is necessary to form morphological elements
% at the same time, these collective structures act at larger scales, so presumably they form appreciable energy sinks, meaning they cannot be described 
% independent of 


So in the intermediate range, between the idealized concepts of the Einstein and Bagnold paradigms, we have a problem of emergent interactions. 
Bedload transport generates and modifies morphological features, but it also responds to them.  
In gravel bed rivers, these feedbacks mix individual motions with the collective formation, motion, and destruction of morphological features, such as bars, bedload sheets, or clusters; and as a result, within the intermediate range, bedload fluxes exhibit variations across multiple temporal and spatial scales, which can be associated with these different processes \citep{Hoey1992, Cudden2003, Nelson2010, Saletti2015, Dhont2018}. 
We can speculate that more advanced physical theories of bedload motion will synthesize Einstein and Bagnold's concepts, with a careful understanding of channel morphology, to understand this coordination between transport and morphology.
There are essentially three components to this synthesis. 
First, the success of Bagnold's paradigm at high transport rates shows us that when motion is dense enough to imitate a continuous flow, the transport is governed by its dissipating effects on the fluid flow; so we need to build deeper understanding of the energetics and dynamics of channel morphology. 
Second, Einstein's paradigm is successful in the opposite extreme. 
When motions are sparse, they result from anomalies in driving or resisting forces, and have a realtively minimal effect on power dissipation in the fluid flow. 
However, these irregular motions spawn morphological features, which can migrate or disintegrate in collective motion, more similar to the paradigm of Bagnold. 
Third, need to synthesize energy approaches with more realistic models based on inidivudal motions
since neither einstein or bagnold aren't too developed, should work on these separately
but they should also be fused 

so no reason to not work on einstein and bagnold separately 
should probably also work on both



Einstein's paradigm is successful in the opposite extreme; when motions are sparse, they result primarily from variations in driving or resisting forces and have a relatively minimal effect on power dissipation in the fluid flow. 
What we really need is a theory containing both elements: an energy budget with provisions for the formation, degradation, and transport of morphological features; and intermittent motion, stemming from variability in driving and resisting forces on the scale of individual grains. 
This synthesis of Einstein and Bagnold might get us quite far toward addressing problems around gravel bed rivers. 




\bibliography{biblio}
\end{document}

