\begin{abstract}

A foundational problem in river science is the calculation of the bedload flux, or the rate of movement of bed material. 
This is a significant problem because bed material movement shapes river channels and determines habitat suitability to a large extent.  
Conceptualizing bedload transport as a random switching between motion and rest states leads naturally to an analogy with ecological population modeling. 
Erosion, movement, and deposition are analogous to birth, migration, and death. 
As in ecology, details of these processes are considered difficult to predict, but their rates can be measured. 
These process analogies have opened up new ways to model bedload transport as a stochastic process.
I review approaches which model the bedload flux as a birth-death process. 
These approaches are exciting because they provide a new theoretical framework for understanding the statistical properties of bedload fluxes -- fluctuations, intermittency, spatial and temporal correlations. 
However, existing approaches contain many simplifying assumptions, so the literature review reveals a handful of obvious opportunities for extensions to these birth-death models. 

The review is used to define a research trajectory, which I extrapolate from in order to discern a set of possible extensions. 
I develop some of these extensions myself, and present (1) a new birth-death model incorporating bed surface affects in sediment entrainment and deposition for multiple size fractions, (2) a joint probabilistic modeling of bed elevation and the bedload flux, and (3) a two-dimensional birth-death theory of bedload motion, where particles are free to move in longitudinal and transverse directions. 
After these extensions are completed, I conclude with a discussion of the scope and limitations of birth-death approaches, and I frame some possible extensions in terms of interdisciplinary literature in physics, chemistry, and ecology. 
The wider directive of this synthetic review is to foster future research in birth-death modelling of bedload transport, because I believe this theoretical framework provides a foundation for deeper understanding of the relationship between channel morphology and bedload fluctuations, with potential to clarify both of these subjects, each of  contemporary relevance to ecology and engineering. 

\end{abstract}


\begin{comment}


Bedload moves intermittently, so it is well suited to describtion by stochastic mathematics.  
In the last several decades, the stochastic approach to modeling bedload transport has been reworked, lending it a firmer mathematical and mechanical foundation.

- the rate of downstream movement of bed material, or the bedload flux, is a foundational and important unsolved problem in river science.
- it has important applications in widespread environmental science
- in the last two decades, stochastic approach has been reworked leading to a firmer mathematical and mechanical foundation 
- Withi
\end{comment} 
