
\subsection{The diffusion of bedload from a microstructural basis}


\citet{Ancey2008} developed a generalized version of \citet{Einstein1950} which took account of a new process to describe the large fluctuations seen in experimental bedload flux signals: collective entrainment. 
\citet{Turowski2009} generalized this model to finite sediment availability, while \citet{Heyman2013} and \citet{Ma2014} used the \citet{Ancey2008} model in its original form to understand the distribution of waiting times between emigration events and the bedload flux probability distribution at a point in space, respectively. 
These two works developed new understanding of spatial and temporal aspects of bedload flux: \citet{Heyman2013} demonstrated that the waiting time between successive emigration events expressed contributions from slow and fast timescales related to individual and collective entrainment, respectively. 
\citet{Ma2014} highlighted the role of observation timescale on the magnitude of flutuations, and discriminated three distinct scaling regimes for bedload fluctuations which are contingent on a dimensionless number $Ra(\delta t)$, where $\delta t$ is the timescale of observation. 

However, up to this point in early 2014, stochastic birth-death models had focused on a single region of space -- a control volume \citep{Einstein1950, Ancey2006, Ancey2008}, or a single point in space -- the downstream end of the control volume, or the plane across which emigration events happen \citep{Heyman2013, Ma2014b}.
This approach is obviously oversimplified. Variability of sediment transport rates is almost a defining feature of them \citep{Hassan2008, Venditti2016, Nelson2014}. 
There is a need to take account of spatial differences in entrainment and deposition characteristics in calculating bedload fluxes. 
Additionally, there are a wide set of issues centered around bedload diffusion-- or the spreading of bedload particles as they are tracked through time and subjected to random entrainment and deposition events \citep{Hassan2016}. Indeed, this study of bedload diffusion was the original research directive of \citet{Einstein1937}. 

Often, bedload diffusion has been understood through deterministic partial differential equation models. These advection diffusion equations have been derived by considering conservation of mass \citep{}, but they were not very well connected to the underlying stochastic dynamics of bedload transport. 
Generating this connection between a microstructural stochastic model like \citet{Ancey2008} and the continuum diffusion equation treatment of bedload diffusion \citep[e.g.][]{Parker2002} was the directive of the next papers on birth-death modeling \citep{Ancey2014,Ancey2015}. 
These papers took a conceptually straightforward extension of the \citet{Ancey2008} model which leads to difficult mathematics. 

Rather than considering a single control volume, as in \citet{Ancey2008}, \citet{Ancey2014,Ancey2015} considered an infinite array of adjacent control volumes (cells), indexed by $i=1,2,\dots,M$. 
Bedload particles can entrain and deposit within each cell, and they can also migrate between cells. 
Each of these transitions are characterized by probabilities per unit time in generalization of \citet{Ancey2008} across spatial extent. 
The $i$th cell has a random number of particles $N_i$ in motion within it. 
Therefore the state of the system at an instant of time is fully characterized by the set of these numbers: $\textbf{n} = (n_1(t), n_2(t), \dots n_M(t))$. 

One target is the grand probability distribution of the number of active particles within each cell at an arbitrary time: $P(\textbf{n};t)$. 
Another target is the continuum limit: letting the length $\Delta x$ of each cell shrink to zero develops an advection-diffusion equation for bedload diffusion. 
Ancey et al termed this a "Stochastic interpretation of bedload diffusion": and the latter target has much more fundamental scope.
It explores the link between microscale stochastic models of bedload transport and mesoscale deterministic models based upon advection-diffusion equations. 

Now I'll sketch the \citet{Ancey2014, Ancey2015} derivation of a birth-death-migration model across an array of cells. 
There are essentially three transitions to take account of: 
\begin{enumerate}
\item Entrainment can occur within each cell ($i$) at probability per unit time $(\lambda_i + \mu_i n_i ) \delta t$
\item Deposition can occur within each cell ($i$) at probability per unit time $ \sigma_i n_i \delta t$
\item Migration can occur from cell $i$ to cell $i+1$ (downstream) at probability per unit time $\nu_i$
\end{enumerate} 
Now some notation is introduced to write the effect of these transitions on $\textbf{n}$ in shorthand. 
If the change of state due to one of these transitions is written as $\Delta \textbf{n}$, then the effect of these transitions can be characterized using vectors $\textbf{r}_i^j$ and $\textbf{r}_i^\pm$. The vectors $ \textbf{r}_i^j$ are of the same dimension as $\textbf{n}$ and all but two of their entries are zero: $r_i = 1$, $r_j = -1$, $r_k=0$ for $k\neq i,j$. The vectors $\textbf{r}_i^\pm$ are the same dimension as $\textbf{n}$ and all but their $i$th entry is zero: $r_i = \pm 1$, $r_j = 0$, $j\neq i$. 

Thus the transition probabilities of each process can be written: 
\begin{enumerate}
\item $ p_i^3 = \text{Prob}(\textbf{N} = \textbf{n} + \textbf{r}_i^+; t+ \delta t) = (\lambda_i + \mu_i N_i) \delta t $ -- entrainment
\item $ p_i^2 = \text{Prob}(\textbf{N} = \textbf{n} + \textbf{r}_i^-; t+\delta t) = \sigma_i N_i \delta t $ -- deposition
\item $ p_i^1 = \text{Prob}(\textbf{N} = \textbf{n} \textbf{n} + \textbf{r}_i^{j-1}; t+ \delta t) = \nu_{i-1}N_{i-1} \delta t $ -- migration
\end{enumerate} 

These transition probabilities develop a master equation analogous to the one from \citet{Ancey2008}, but generalized to a collection of $M$ cells: 
\begin{multline} \frac{\partial}{\partial t} P(\textbf{n};t) = \sum_{i=1}^M (n_i+1)[P(\textbf{n}+\textbf{r}_{i+1}^j,t) \nu_i + P(\textbf{n}+\textbf{r}_i^+,t)\sigma_i ]\\
+ P(\textbf{n}+\textbf{r}_i^-,t)(\lambda_i + \mu_i(n_i-1))\\
+ P(\textbf{n}+\textbf{r}_i^{j-1},t)\nu_{i-1}n_{i-1}\\
- P(\textbf{n},t)(\nu_{i-1} n_{i-1}+\lambda_i + \mu_i n_{i+1} + \nu_i n_i + \sigma_i n_i ) \end{multline} 
Information about solving this equation is sparse in the literature. 
The closest approach has been in ecology: these birth/death/migration-type stochastic models have been considered in context of ecological population dynamics since at least the sixties \citep{Bailey1968}. 
One exact solution of a very similar stochastic model has been attained using path integral approaches borrowed from quantum field theory \citep{Field2010}.
By any metric, the analytical barrier this multidimensional master equation presents is not trivial, although definitely numerical solutions are possible \citep[e.g.][]{Gillespie1992}. 

In order to link this equation to spatial diffusion, \citet{Ancey2014, Ancey2015} resorted to a Poisson space representation \citep{Gardiner1983}. 
In effect, the Poisson transformation converts the discrete random variable $\textbf{N}$ to a continuous random variable $\textbf{a}$ (the Poisson rate). 
The Poisson transform is 
\be P(\textbf{n},t) = \prod_i\int \frac{e^{-a_i}a_i^n}{n!}f(\textbf{a},t)d\textbf{a}, \ee
where $\textbf{a} = (a_i) \geq 0$ for $i=1,2,\dots$ and $f(\textbf{a},t)$ is the multivariable probability density of the continuous vector $\textbf{a}$. 

\citet{Ancey2014, Ancey2015} used this transformation to obtain a Langevin equation representation of the dynamics of the random variable $\textbf{a}$.
That is, they obtained a stochastic differential equation representation of $\textbf{a}$: 
\be da_i(t) = (\lambda_i - a_i(\sigma_i-\mu_i) + \nu_{i-1} a_{i-1} - \nu_i a_i)dt + \sqrt{2 \mu_i a_i}dW_i(t),\ee
and the $W_i(t)$ is a white noise term (Wiener Process) on cell $i$. 
The mean steady state solution of this Langevin equation is 
\be \bra a \ket_ss = \frac{\lambda}{\sigma-\mu} \ee 
for all $i$. 

Now \citet{Ancey2014,Ancey2015} sought out to find an advection diffusion equation as the limit of cell size $\Delta x \rightarrow 0$, so they went on to impose this limit on the Langevin equation, thereby obtained an advection diffusion equation. 
Unfortunately, the limit is more subtle than expected. 
This did not really work for reasons I need to understand better, and it did not work in ways I need to learn to explain better. 
More study is needed on spatially varying birth-death models, from me, and from others especially.



