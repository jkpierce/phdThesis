\subsection{What about sediment availability?: Turowski's two species model} 

Semi-alluvial channels, where alluvium sits on top of bedrock, will violate one of the key assumptions of the \citet{Ancey2008} framework. 
This model implicitly assumes that the bed surface has an infinite number of particles available for entrainment. 
In reality, this situation will never be satisfied, even in alluvial channels, because there are a wide variety of stabilizing processes acting on bed surfaces \citep{Hassan2008, Venditti2016}. 

Turowski was concerned with bedrock scour due to alluvium. 
Therefore he extended the Ancey framework. 
The Ancey framework takes account of one population: the number of moving particles. 
Turowski's extension included a second population into the framework: the number of particles available for entrainment. 
He considered that as a particle entrains into a motion state, the number of particles available for entrainment decreases, and likewise as a particle deposits from the motion state, the number of particles available for entrainment increases. 
Therefore, he considered a probabilisitic population model of two coupled populations: such an inquiry is entirely new in stochastic theory of the bedload flux, although it is commonplace in ecological modeling \citep[e.g.][]{Pielou1977}

Turowski considered a joint probability distribution for the random number $N$ of moving particles and $M$ of particles available for entrainment. 
This can be written $\pi_{n,m}(t)$. 
Within a small time interval $dt$ he considered birth, death, immigration, and emigration transitions governing the probabilities $\pi_{n,m}(t)$ similar to \citet{Ancey2008}. 
Like \citet{Ancey2008}, he also included a collective entrainment contribution, arranging that the model expressed wide Negative-Binomial-like fluctuations in the bedload rate. 

Turowski's transition probabilities in $dt$ varied by including the number of available particles $m$ into the entrainment and deposition probabilities. He considered a rate of entrainment $(\lambda_0 + \mu n)m$: when the population of particles available for entrainment is zero ($m=0$), the rate of entrainment is zero. The entrainment process, when it occurs, enacts a change of state $(n,m)\rightarrow(n+1,m-1)$: the number of active particles increases by one, while the number of particles available for entrainment decreases by one. 

Similarly, Turowski's deposition process was generalized from \citet{Ancey2008}.
Deposition occurs with rate $\sigma n$ in time $dt$, just as in Ancey et al, but when this deposition process occurs it enacts the change $(n,m) \rightarrow (n-1,m+1)$ in the coupled populations. 
The Kolmogorov equation is 
\begin{multline}
P(m,n;t+dt) = P(m,n-1,t)\lambda_0 dt + P(m+1,n-1,t)((m+1)\lambda_1 \\+ (m+1)(n-1)\mu)dt + P(m-1,n+1,t)(n+1)\sigma dt + P(m,n+1,t)(n+1)\nu dt \\+ P(m,n,t)*[1-dt(\lambda_0 + m \lambda_1 + n \nu + n \sigma + mn \mu). \end{multline}
which in the limit $dt \rightarrow 0 $ becomes a Master equation for the joint probabilities $\pi_{n,m}(t)$: 
\begin{multline} 
\frac{d}{dt} P (m,n,t) = \lambda_0 P(m,n-1,t) +\\ ((m+1)\lambda_1 + (m+1)(n-1)\mu)P(m+1,n-1,t) + (n+1)\sigma P(m-1,n+1,t)\\ + (n+1)\nu P(m,n+1,t) - (\lambda_0 + m \lambda_1 + n\nu + n \sigma + mn \mu ) P(m,n,t). 
\end{multline} 

Again, as a reminder, the number of active particles is $n$ and the number of particles available for entrainment is $m$: this master equation describes a hirearchy of joint probability distributions for these discrete random varaibles. 
Exact solutions via counting arguments \citep[e.g.][]{Ancey2006} or probability generating functions \citep[e.g.][]{Ancey2008} are evidently no longer tractable. 
\citet{Turowski2009} resorted to a numerical solution.
Numerical algorithms for stochastic birth-death models are relatively easy to implement and are well described in a number of references \citep[e.g.][]{Gillespie1977, Gillespie1992}. 
Turowski pursued one of these algorithms and related his transport rate computations to bedrock erosion and alluvial cover within a bedrock channel. 

Now we briefly describe the stochastic simulation algorithm of Gillespie 1977 ... 
Then we reproduce Turowski's results using it ... 

We also explore the possibility of an exact solution for the Turowski model.. go harder than he did. 

Turowski's master equation reduces to the Ancey et al result as $m \rightarrow \infty$. As $m$ gets very large, $m-1 \approx m \approx m+1$: effectively, large $m$ makes $m$ constant as far as the master equation perceives. 
In this limit the probability $P(m,n,t) \approx P(n,t)$, and since $m$ is effectively constant, one can take $m\lambda_1 \rightarrow \lambda_1'$ and $m\mu \rightarrow \mu'.$ 
With these substitutions the master equation reproduces the \citet{Ancey2008} result. 
For our purposes, given our concern with alluvial channels as outlined in the introduction, the main takeaway of Turowski's work is that the birth-death modelling approach can be applied to study the stochastic dynamics of multiple populations whose variations are correlated. 
This is similar to competing species in ecology \citep{Pielou1977} or to multiple reacting chemical species \citep{Gardiner1983}. 

