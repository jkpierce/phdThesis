%% The following is a directive for TeXShop to indicate the main file
%%!TEX root = diss.tex

\chapter{Preface}

This thesis is entirely the original research of Kevin Pierce, including all figures, writing, and calculations.
The supervisory committee and especially the research supervisor Marwan Hassan provided guidance in the research.
The thesis includes two published works, Chapters \ref{ch:ch3} and \ref{ch:downDiff}, available in \textit{Journal of Geophysical Research: Earth Surface} and \textit{Geophysical Research Letters} respectively, with Marwan Hassan as coauthor \citep{Pierce2020a,Pierce2020b}. Chapters \ref{ch:flux} and \ref{ch:langevin} will be submitted for publication after acceptance of the thesis.


\endinput
The 


This thesis deals with the transport of coarse sediment in flowing water from the perspective it fundamentally results from the movements of individual grains.
The research has been motivated by the belief that geomorphology as a science will benefit from mechanistic, process-based, mathematical models of sediment transport processes.

Geomorphology has long been characterized by observation, description, and the building of conceptual models, not mathematical ones.
Only relatively recently has the science turned toward quantitative methods, with researchers working to frame observations in terms of underlying processes and to describe them with methods adapted from physics.

Given the complexity of geomorphology problems, a complete reduction of geomorphology to physics would be narrow-minded, but we can nonetheless borrow ways of thinking.
An extremely successful approach in physics has been to construct idealized models with little intention of direct realism, study them deeply to obtain complete understanding, and then, decades or centuries later, to embellish these models with more sophisticated features to describe real-world phenomena.
An example is the block on the spring, the simple harmonic oscillator, which starts as a toy model studied by every first year physics student, but somehow shows up in the deepest inquiries of theoretical physics, from quantum matter to cosmological inflation.
This thesis applies this ``simple harmonic oscillator approach" to some problems in Earth science.
As a result, there are big assumptions: sediment grains all have the same size, and water flows are constant through time. 
The objective is to build up simple archetypes which can be embellished later, not to build \textbf{the model} of fluvial geomorphology. 
That's for later! (And not for me.) CHEERS.
