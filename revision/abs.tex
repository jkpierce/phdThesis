%% The following is a directive for TeXShop to indicate the main file
%%!TEX root = diss.tex

\chapter{Abstract}

Bedload transport is the movement of coarse grains through river channels by bouncing, rolling, and sliding.Because coarse grains control river stability, predicting the rate of bedload transport is a fundamental problem in river science. This problem is usually approached with continuum mechanics, but this approach is questionable considering that coarse sediment grains rarely move in densities approximating a continuum.

An alternative approach describes bedload transport from the trajectories of individual grains using statistical physics. This approach has become increasingly popular in recent decades, but many fundamental issues prevent this approach from being widely adopted. In particular, the connection between individual particle trajectories and transport rates remains unclear, and particle trajectory models remain highly simplified. Feedbacks between topography and sediment transport remain challenging to analyze, and basic properties of bedload motions like downstream travel velocities remain incompletely understood. Buried particles cannot appreciably move downstream, but even this simple observation has not been comprehensively described in the statistical physics approach.

This thesis presents four projects completed in my PhD which overcome these issues to provide new understanding of bedload transport from a statistical point of view. First, I demonstrate how to calculate the sediment flux from the dynamics of individual grains, and I model the trajectories of grains alternating through motion and rest having fluctuating velocities in the motion state. This links the sediment flux to the grain-scale dynamics and describes particle trajectories with additional detail compared to earlier descriptions. Second, I include feedbacks between local bed elevations and sediment transport, quantifying the interplay of bed elevation changes and sediment transport rates and predicting how long particles can stay buried in the river bed. Third, I incorporate the sediment burial process into a model of downstream sediment transport, predicting how grains move downstream when they can become buried. Finally, I describe bedload dynamics on short timescales, predicting the movement velocities of bedload particles using methods adopted from granular physics. I conclude by summarizing these developments, discussing their implications for the statistical description of bedload transport, and suggesting how we can use this modeling progress to better understand landscapes.

