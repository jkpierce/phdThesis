%% The following is a directive for TeXShop to indicate the main file
%%!TEX root = diss.tex

\chapter{Abstract}

Bedload transport is the movement of coarse sediment grains in a river channel by rolling, sliding, and bouncing.
Motivated by the need to predict river morphodynamics, a central problem in river science is to calculate the downstream flux of grains moving as bedload.
This problem is usually formulated in terms of continuum mechanics, but this approach is questionable considering that bedload grains are coarse, they rarely reach densities approximating a continuum.
This thesis contains four projects that abandon the continuum hypothesis to interpret sediment transport in terms of individual grains.
First, I develop a new theory to calculate the sediment flux from the trajectories of individual grains as they alternate between motion and rest.
This leads to a sediment flux probability distribution which predicts mean sediment fluxes that depend on the timescale over which they are observed, explaining why measurements of the bedload flux can give such poor results.
Second, I evaluate the interplay between bed elevation changes and sediment transport rates, finding that because aggradation and degradation oppose changes in the sediment flux, bed elevation changes buffer sediment transport fluctuations.
Third, I incorporate the process of sediment burial into a model of downstream sediment trajectories, demonstrating that burial eventually halts downstream transport.
This has implications for the transport of contaminants in river channels.
Finally, using concepts borrowed from the theory of granular gases, I develop a stochastic model of individual particles moving downstream in the turbulent flow through a sequence of collisions with the bed surface. This indicates that the velocity characteristics of sediment particles are tuned by the amount of dissipation particles experience when colliding with the sedimentary bed.
This research has implications across river science, where there is increasing acknowledgement of the shortcomings of the continuum hypothesis for predicting dynamics in mountain streams. This work extends a wide body of work on individual particle motions which originates from the 1930s, and unifies it with more recent approaches to calculate the bulk bedload flux which have not, until now, been related to the movement characteristics of individual grains.

\endinput

A central task within Earth science is to understand and predict the evolution of landscapes due to flowing water.
As rainfall channelizes and flows downhill, it carves out basins and etches in networks of intersecting channels.
In the highlands of these networks, water arranges boulders and gravels into an intricate array of patterns -- steps, pools, bars, and riffles, which cooperate with vegetation both live and dead to set the stage for much of Earth's biota. In the lowlands, channels laden with old mountain sediments, weathered and abraded to fine particles, splay across floodplains and drift between configurations through millennia.

Sediment transport is a main driver of all fluvial dynamics.
Channel evolution ultimately occurs because individual sediment grains move from one location to another. Yet in a majority of modelling studies, landscapes are represented as continua, where the locations of individual grains are averaged away, and sediment transport is represented as a steady stream of mass, rather than the intermittent movements of individual grains
Useful as this continuum approach may be, many fluvial phenomena are not well-suited for it.
Sediment transport rates are known to fluctuate widely through space and time, and these fluctuations are understood to initiate bedform development and control channel widths. The largest grains in small mountain channels are known to confer the most stability, while violating any assumption of being small compared to the scales of interest. To understand fluvial dynamics, we need the capability to model discrete grains, not just continua.

In this thesis, I present four or five years of my theoretical research into the movements of individual sediment grains in river channels.
This sub-field of Earth science, with a focus on grain-scale process, has existed at least since Hans Albert Einstein in 1937, and within it, modellers have traditionally compromised on severe approximations to balance realism against mathematical difficulty.
In this enterprise, perfectly flat beds which do not change shape, spherical grains, infinite movement velocities, and turbulence free flows have been typical assumptions to make progress, even though they have little basis in reality.
The research presented here takes on more mathematical difficulty than before to introduce more realism, adding some bricks to the fortress, and introducing new methods to the field which should work well for other people to make more bricks later. I hope you find it useful! Happy reading.
