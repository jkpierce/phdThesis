%% The following is a directive for TeXShop to indicate the main file
%%!TEX root = diss.tex

\chapter{Preface: Geomorphology traditions}
Geomorphology can be defined as the study of Earth's evolution by tectonics, weathering, and erosion. 
This science has long been characterized by observation and description. Only relatively recently did the science turn to quantitative methods, with researchers working to frame observations in terms of underlying processes and to describe them with mathematical models modified from physics \citep{Church2005}.

The overreliance on physics methods in geomorphology has been rightly criticized as overly reductionist \citep{Slaymaker2020}.
In physics, many of the greatest achievements describe closed systems with perfect order, but these idealized systems stand in contrast to those encountered in geomorphology.
Earth's surface is open, and it is highly disordered.
The surface receives fluxes of mass and energy across its boundaries by tectonics, climate, transpiration, and sunlight, and its component parts vary widely in their characteristics, activities, and compositions.
Earth's problems concern biota, society, and the deepest issues in the human sphere.
With such complexity, a pure reduction of geomorphology to physics is not a goal to work toward.

But we can borrow ways of thinking.
An extremely successful approach in physics has been to construct idealized models without any intention of realism, study them deeply to obtain complete understanding, and then, decades or centuries later, to embellish these models with more sophisticated features to describe real-world phenomena.
The example which comes to mind is the block on the spring, the simple harmonic oscillator, which starts as a toy model studied by every first year physics student, but somehow ends up fundamental in the deepest inquiries of theoretical physics from superconductivity to cosmological inflation \citep{Fetter2003,Liddle2000}.
This thesis applies this start dumb approach, inspired by physics, to some problems in Earth science.
The target is to build archetypes, not describe the Earth's surface in all its grandeur. 
That's for later (and not for me!)

