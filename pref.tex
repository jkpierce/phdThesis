%% The following is a directive for TeXShop to indicate the main file
%%!TEX root = diss.tex

\chapter{Preface: Complementary traditions}

This thesis deals with the transport of coarse sediment in flowing water from the perspective that sediment transport results from the movements of individual grains.
The research has been motivated by the belief that geomorphology as a science will benefit from mechanistic, process-based, mathematical models of sediment transport processes.

Geomorphology has long been characterized by observation, description, and the building of conceptual models, not mathematical ones.
Only relatively recently has the science turned toward quantitative methods, with researchers working to frame observations in terms of underlying processes and to describe them with methods adapted from physics.

An over-reliance on physics methods in geomorphology can be criticized as simplistic and reductionist.
After all, in physics, many of the greatest achievements describe closed systems with perfect order, and these stand in contrast to those encountered in Earth's landscapes.
Earth's surface is open, and it is highly disordered.
It receives fluxes of mass and energy across its boundaries by tectonics, climate, transpiration, and the sun, and its component parts vary widely in their characteristics, activities, and compositions.

With such complexity, a complete reduction of geomorphology to physics is unlikely, but we can borrow ways of thinking.
An extremely successful approach in physics has been to construct idealized models with little intention of direct realism, study them deeply to obtain complete understanding, and then, decades or centuries later, to embellish these models with more sophisticated features to describe real-world phenomena.
An example is the block on the spring, the simple harmonic oscillator, which starts as a toy model studied by every first year physics student, but somehow ends up incorporated in the deepest inquiries of theoretical physics, from quantum matter to cosmological inflation.
This thesis applies this start simple approach to some problems in Earth science.
The objective is to build up simple archetypical models which can be embellished later, not to solve \textbf{the model} of fluvial geomorphology. 
That's for later (and not for me!) CHEERS.
