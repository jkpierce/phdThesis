%%!TEX root = ../diss.tex

\chapter{Summary and future work}
\label{ch:conc}
\section{Overall Methodology}

This thesis has presented a review and four chapters which generalize and extend earlier work to describe sediment transport as a stochastic process. In the research, I have characterized the movements of grains along a wider range of timescales than before, connected a wide literature of earlier research by framing it as different applications of Langevin and Master equations, and extended these earlier works to link  



\subsection{Langevin and master equations}

Probability 


\subsection{Idealized noises and their combinations}


\section{Key contributions}

\subsection{Probability distribution of the sediment flux from micromechanics of particle trajectories}

\subsection{Inclusion of velocity fluctuations into Einstein's model of individual particle trajectories}


\subsection{Quantification of the control of bed elevation fluctuations over sediment transport fluctuations}


\subsection{Predicting how sediment burial affects the downstream spreading of sediment tracer particles}


\section{Conclusion}

This thesis embraced variability in grain scale transport to form some new descriptions of sediment transport processes in river channels.
The remaining question is how variability in sediment transport impacts landscapes.

The founders of process-based geomorphology generally accepted the role of variability in geomorphology \citep{Horton1945,Strahler1952,Langbein1964}, although their main efforts were to develop strategies to describe landscapes despite it, including competent conditions to summarize climatic fluctuations \citep{Wolman1959,Wolman1978}, characteristic grain sizes to overcome sorting processes \citep{Parker1982,Andrews1983}, and averaged flow conditions to avoid turbulence \citep{MeyerPeter1948,Bagnold1954}.

The stochastic models of sediment transport developed in this thesis suggest a methodology we can use to step beyond averaged descriptions of landscape evolution, propagate noises through the governing equations, and revisit this old question of how variability shapes Earth's surface.